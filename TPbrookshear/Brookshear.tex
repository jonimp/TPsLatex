\documentclass[a4paper,12pt]{article}
\usepackage[a4paper, top=2.5cm, bottom=2cm, left=2cm, right=2cm]{geometry}
\usepackage[utf8]{inputenc}
\usepackage{fancyhdr}
\usepackage{fancyvrb}
\usepackage{enumitem}
\usepackage{xparse}
\usepackage{graphicx}
\usepackage{xcolor}

\pagestyle{fancy}
\fancyhf{}
\rhead{INSPT - UTN}
\lhead{Jonatan Imperi}
\cfoot{\thepage}

\begin{document}
	
	\begin{center}
		
		\LARGE \textbf{Sistemas de computación 1} \\[0.5cm]
		\LARGE Ejercitación máquina de Brookshear \\
	\end{center}
	
	\vspace{1cm}
	
	1) \hangindent=3em Llenar las últimas 10 direcciones de la memoria con los primeros 10 múltiplos naturales del contenido de la	posición 02, si éste es natural.

	\vspace{0.5cm}

	\renewcommand{\arraystretch}{1.1}
	\begin{tabular}{l|c|l}
		\textbf{Pos. memoria} &\textbf{Instrucción} & \textbf{Detalle}\\
		\hline
		00 / 01 & B004  & Salta a la instrucción 04 \\ 
		02 / 03 & 02F6  & Contiene el dato a tratar \\
		04 / 05 & 2101  & Se cargó el registro 1 con el valor 01 \\
		06 / 07 & 22FF  & Se cargó el registro 2 con el valor FF (-1) \\
		08 / 09 & 1302  & Se cargó R3 con el contenido de la celda de memoria 02 \\
		0A / 0B & 1403  & Se cargó R4 con el contenido de la celda de memoria 03 \\
		0C / 0D & 2509  & Se cargó el registro 5 con el valor 09\\
		0E / 0F & 5636  & Se suman los R3 y R6 guardandose en R6 \\
		10 / 11 & 36F6  & Se copia el contenido de R6 en la celda F6 \\
		12 / 13 & 5414  & Se suman R1 y R4 guardandose en R4 \\
		14 / 15 & 3411  & Se copia el contenido de R4 en la celda 11 \\
		16 / 17 & B51C  & Compara R0 con R5, si son iguales (a 0), salta a 1C \\
		18 / 19 & 5525  & Se suman R2 y R5 guardandose en R5 \\
		1A / 1B & B00E  & Salta a la instrucción 0E \\
		1C / 1D & C000  & Fin del programa \\
	\end{tabular}

\end{document}