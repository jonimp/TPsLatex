\documentclass[a4paper,12pt]{article}
\usepackage[utf8]{inputenc}
\usepackage{amsmath}
\usepackage{listings}
\usepackage{fancyhdr}
\usepackage{fancyvrb}
\setlength{\headheight}{15pt}
\usepackage{caption}
\usepackage{graphicx}
\usepackage{xcolor}
\usepackage{xparse}
\usepackage{enumitem}
\usepackage{float}
\usepackage[utf8]{inputenc}
\usepackage{textcomp}
\usepackage[a4paper, top=2.5cm, bottom=2cm, left=2cm, right=2cm]{geometry}

\pagestyle{fancy}
\fancyhf{}
\rhead{INSPT - UTN}
\lhead{Jonatan Imperi}
\cfoot{\thepage}
\renewcommand{\theenumi}{\alph{enumi}}


\begin{document}
	
	\begin{center}
		
		\LARGE \textbf{Sistemas de computación 1} \\[0.5cm]
		\LARGE Trabajo práctico n° 3 \\
	\end{center}
	
	\section*{Formato normalizado IEEE754 para coma flotande de 32 bits.}
	
	1.Obtener la representación del número decimal en el formato normalizado IEEE754 para coma flotante de 32 bits, de los siguientes números:
	\begin{enumerate}
		\item -0.00015
		\item 53.2874
		\item 291.072
		\item -6.2265625
		\item 14
		\item 3.5
		\item -12.5
		\item 10.25
		\item -6.75	
	\end{enumerate}	

	\subsection*{Método de conversión}
	Este formato conformado por 1 bit para el signo, 8 bits para el exponente y 23 bits para la mantisa, se empieza a componer por el primer bit considerando 1 si es negativo y 0 si es positivo. El siguiente paso es representar el número decimal en binario natural para luego trasladar la coma detrás del primer 1 contando cuantos bits se movieron. Este traslado puede ser tanto para la izquierda como para la derecha, según donde se encuentre el primer 1 partiendo desde la izquierda. Si en el traslado se hace para la izquierda, el número será positivo, en cambio si se trasladara hacia la derecha, el número será negativo. 
	
		\begin{center}	
		\colorbox{yellow}{{\textbf{a.} $-0.00015$}}
		
		\subsection*{Bit de signo}
		
		\[
		\text{El número es negativo} \Rightarrow \text{bit de signo = } \boxed{1}
		\]
		
		\subsection*{Conversión a binario}
		
		
	~~~~~~~~~~~~~~~~~~~~~~~~~\text{Parte entera: 0 → 0} 
		\newline
		
		\text{Parte decimal: Multiplicamos sucesivamente por 2}
		
		\begin{center}
		\begin{Verbatim}
			0.00015 * 2 = 0.00030 → 0 
			0.00030 * 2 = 0.00060 → 0 
			0.00060 * 2 = 0.00120 → 0 
			0.00120 * 2 = 0.00240 → 0 
			0.00240 * 2 = 0.00480 → 0 
			0.00480 * 2 = 0.00960 → 0 
			0.00960 * 2 = 0.01920 → 0 
			0.01920 * 2 = 0.03840 → 0 
			0.03840 * 2 = 0.07680 → 0 
			0.07680 * 2 = 0.15360 → 0 
			0.15360 * 2 = 0.30720 → 0 
			0.30720 * 2 = 0.61440 → 0 
			0.61440 * 2 = 1.22880 → 1 
			0.22880 * 2 = 0.45760 → 0 
			0.45760 * 2 = 0.91520 → 0 
			0.91520 * 2 = 1.83040 → 1 
			0.83040 * 2 = 1.66080 → 1 
			0.66080 * 2 = 1.32160 → 1 
			0.32160 * 2 = 0.64320 → 0 
			0.64320 * 2 = 1.28640 → 1 
			0.28640 * 2 = 0.57280 → 0 
			0.57280 * 2 = 1.14560 → 1 
			0.14560 * 2 = 0.29120 → 0 
			0.29120 * 2 = 0.58240 → 0 
			0.58240 * 2 = 1.16480 → 1 
			0.16480 * 2 = 0.32960 → 0 
			0.32960 * 2 = 0.65920 → 0 
			0.65920 * 2 = 1.31840 → 1 
			0.31840 * 2 = 0.63680 → 0 
			0.63680 * 2 = 1.27360 → 1 
			0.27360 * 2 = 0.54720 → 0 
			0.54720 * 2 = 1.09440 → 1 
			0.09440 * 2 = 0.18880 → 0 
			0.18880 * 2 = 0.37760 → 0 
			0.37760 * 2 = 0.75520 → 0 
			0.75520 * 2 = 1.51040 → 1 
			
						
		\end{Verbatim}
		\end{center}
	
		Resultado binario: 
		\[
		-0.00015_{10} \approx -0.000000000000100111010100100101010010
		\]
		
		\subsection*{Determinación del exponente}
		
		Se debe aplicar el sesgo de 127 moviendo 13 lugares a la derecha hasta el primer 1
		\vspace{-0.5em}
		\[
		\text{Exponente real} = -13 \quad \Rightarrow \quad \text{Exponente IEEE} = -13 + 127 = 114
		\]
		\vspace{-0.5em}
		\[
		114_{10} = 01110010_2
		\]
		
		\subsection*{Mantisa (23 bits)}
		
		Se toma lo que sigue después del primer 1 en la forma normalizada:
		\vspace{-0.5em}
		\[
		\text{Mantisa: } \underline{1}.00111010100... \Rightarrow 00111010100100101010010
		\]
		
		\subsection*{Resultado IEEE 754 (formato 32 bits)}
		
		\[
		\text{IEEE 754:} \quad 
		\boxed{1} \quad \boxed{01110010} \quad \boxed{00111010100100101010010}
		\]

		Separado en grupos de 4 bits:
		\vspace{-0.5em}
		\[
		\texttt{1011\ 1001\ 0001\ 1101\ 0100\ 1001\ 0101\ 0010}
		\]
		
		---	
		\end{center}
	%bbbbbbbbbbbbbbbbbbbbbbbbbbbbbbbbbbbbbbbbbbbbbbbbbbbbbbbbbbbbbbbbbbbbbbbbbbbbbbbbbb
	\begin{center}	
		\colorbox{yellow}{{\textbf{b.} $53.2874$}}
		
		\subsection*{Bit de signo}
		
		\[
		\text{El número es positivo} \Rightarrow \text{bit de signo = } \boxed{0}
		\]
		
		\subsection*{Conversión a binario}
				~~~~~~~\text{Parte entera: Dividimos sucesivamente por 2}
		\begin{center}
			\begin{Verbatim}[formatcom=\centering]
				53 / 2 = 26, residuo 1
				26 / 2 = 13, residuo 0
				13 / 2 = 6,  residuo 1
				6  / 2 = 3,  residuo 0
				3  / 2 = 1,  residuo 1
				1  / 2 = 0,  residuo 1
				
				Lectura inversa: 110101
			\end{Verbatim}
		\end{center}
 
		
		~~~~~~~~~\text{Parte decimal: Multiplicamos sucesivamente por 2}
		
		\begin{center}
			\begin{Verbatim}
			bit1: 0.28740 * 2 = 0.57480 → 0 
			bit2: 0.57480 * 2 = 1.14960 → 1 
			bit3: 0.14960 * 2 = 0.29920 → 0 
			bit4: 0.29920 * 2 = 0.59840 → 0 
			bit5: 0.59840 * 2 = 1.19680 → 1 
			bit6: 0.19680 * 2 = 0.39360 → 0 
			bit7: 0.39360 * 2 = 0.78720 → 0 
			bit8: 0.78720 * 2 = 1.57440 → 1 
			bit9: 0.57440 * 2 = 1.14880 → 1 
			bit10: 0.14880 * 2 = 0.29760 → 0 
			bit11: 0.29760 * 2 = 0.59520 → 0 
			bit12: 0.59520 * 2 = 1.19040 → 1 
			bit13: 0.19040 * 2 = 0.38080 → 0 
			bit14: 0.38080 * 2 = 0.76160 → 0 
			bit15: 0.76160 * 2 = 1.52320 → 1 
			bit16: 0.52320 * 2 = 1.04640 → 1 
			bit17: 0.04640 * 2 = 0.09280 → 0 
			bit18: 0.09280 * 2 = 0.18560 → 0 
			bit19: 0.18560 * 2 = 0.37120 → 0 
			bit20: 0.37120 * 2 = 0.74240 → 0 
			bit21: 0.74240 * 2 = 1.48480 → 1 
			bit22: 0.48480 * 2 = 0.96960 → 0 
			bit23: 0.96960 * 2 = 1.93920 → 1 
			bit24: 0.93920 * 2 = 1.87840 → 1 
			bit25: 0.87840 * 2 = 1.75680 → 1 
			bit26: 0.75680 * 2 = 1.51360 → 1 
			bit27: 0.51360 * 2 = 1.02720 → 1 
			bit28: 0.02720 * 2 = 0.05440 → 0 
			bit29: 0.05440 * 2 = 0.10880 → 0 
			bit30: 0.10880 * 2 = 0.21760 → 0 
			bit31: 0.21760 * 2 = 0.43520 → 0 
			bit32: 0.43520 * 2 = 0.87040 → 0 
			bit33: 0.87040 * 2 = 1.74080 → 1 
			bit34: 0.74080 * 2 = 1.48160 → 1 	
			\end{Verbatim}
		\end{center}
		
		Resultado binario: 
		\[
		53.2874_{10} \approx 110101.01001001100100110000101
		\]
		
		\subsection*{Determinación del exponente}
		
		Se debe aplicar el sesgo de 127 moviendo 5 lugares a la izquierda hasta el primer 1
		\vspace{-0.5em}
		\[
		\text{Exponente real} = 5 \quad \Rightarrow \quad \text{Exponente IEEE} = 5 + 127 = 132
		\]
		\vspace{-0.5em}
		\[
		132_{10} = 10000100_2
		\]
		
		\subsection*{Mantisa (23 bits)}
		
		Se toma lo que sigue después del primer 1 en la forma normalizada:
		\vspace{-0.5em}
		\[
		\text{Mantisa: } \underline{1}.1010101001001... \Rightarrow 10101010010011001001100
		\]
		
		\subsection*{Resultado IEEE 754 (formato 32 bits)}
		
		\[
		\text{IEEE 754:} \quad 
		\boxed{0} \quad \boxed{10000100} \quad \boxed{10101010010011001001100}
		\]
		
		Separado en grupos de 4 bits:
		\vspace{-0.5em}
		\[
		\texttt{0100\ 0010\ 0101\ 0101\ 0010\ 0110\ 0100\ 1100}
		\]
		
		---	
	\end{center}
	%ccccccccccccccccccccccccccccccccccccccccccccccccccccccccccccccccccccccccccccccccccccccc
		\begin{center}	
		\colorbox{yellow}{{\textbf{c.} $291.072$}}
		
		\subsection*{Bit de signo}
		
		\[
		\text{El número es positivo} \Rightarrow \text{bit de signo = } \boxed{0}
		\]
		
		\subsection*{Conversión a binario}
		~~~~~~~\text{Parte entera: Dividimos sucesivamente por 2}
		\begin{center}
			\begin{Verbatim}[formatcom=\centering]
				291 / 2 = 145, residuo 1
				145 / 2 =  72, residuo 1
				 72 / 2 =  36, residuo 0
				 36 / 2 =  18, residuo 0
				 18 / 2 =   9, residuo 0
				  9 / 2 =   4, residuo 1
				  4 / 2 =   2, residuo 0
				  2 / 2 =   1, residuo 0
				  1               		
				Lectura inversa: 100100011
			\end{Verbatim}
		\end{center}
		
		
		~~~~~~~~~\text{Parte decimal: Multiplicamos sucesivamente por 2}
		
		\begin{center}
			\begin{Verbatim}
				0.07200 * 2 = 0.14400 → 0 
				0.14400 * 2 = 0.28800 → 0 
				0.28800 * 2 = 0.57600 → 0 
				0.57600 * 2 = 1.15200 → 1 
				0.15200 * 2 = 0.30400 → 0 
				0.30400 * 2 = 0.60800 → 0 
				0.60800 * 2 = 1.21600 → 1 
				0.21600 * 2 = 0.43200 → 0 
				0.43200 * 2 = 0.86400 → 0 
				0.86400 * 2 = 1.72800 → 1 
				0.72800 * 2 = 1.45600 → 1 
				0.45600 * 2 = 0.91200 → 0 
				0.91200 * 2 = 1.82400 → 1 
				0.82400 * 2 = 1.64800 → 1 
				0.64800 * 2 = 1.29600 → 1 
				0.29600 * 2 = 0.59200 → 0 
				0.59200 * 2 = 1.18400 → 1 
				0.18400 * 2 = 0.36800 → 0 
				0.36800 * 2 = 0.73600 → 0 
				0.73600 * 2 = 1.47200 → 1 
				0.47200 * 2 = 0.94400 → 0 
				0.94400 * 2 = 1.88800 → 1 
				0.88800 * 2 = 1.77600 → 1 
				0.77600 * 2 = 1.55200 → 1 
				0.55200 * 2 = 1.10400 → 1 
				0.10400 * 2 = 0.20800 → 0 
				0.20800 * 2 = 0.41600 → 0 
				0.41600 * 2 = 0.83200 → 0 
				0.83200 * 2 = 1.66400 → 1 
				0.66400 * 2 = 1.32800 → 1 
				0.32800 * 2 = 0.65600 → 0 
				0.65600 * 2 = 1.31200 → 1 
				0.31200 * 2 = 0.62400 → 0 
				0.62400 * 2 = 1.24800 → 1 
				0.24800 * 2 = 0.49600 → 0  	
			\end{Verbatim}
		\end{center}
		
		Resultado binario: 
		\[
		53.2874_{10} \approx 100100011.000100100110111010010111100011010
		\]
		
		\subsection*{Determinación del exponente}
		
		Se debe aplicar el sesgo de 127 moviendo 8 lugares a la izquierda hasta el primer 1
		\vspace{-0.5em}
		\[
		\text{Exponente real} = 8 \quad \Rightarrow \quad \text{Exponente IEEE} = 8 + 127 = 135
		\]
		\vspace{-0.5em}
		\[
		135_{10} = 10000111_2
		\]
		
		\subsection*{Mantisa (23 bits)}
		
		Se toma lo que sigue después del primer 1 en la forma normalizada:
		\vspace{-0.5em}
		\[
		\text{Mantisa: } \underline{1}.00100011000100... \Rightarrow 00100011000100100110111
		\]
		
		\subsection*{Resultado IEEE 754 (formato 32 bits)}
		
		\[
		\text{IEEE 754:} \quad 
		\boxed{0} \quad \boxed{10000111} \quad \boxed{00100011000100100110111}
		\]
		
		Separado en grupos de 4 bits:
		\vspace{-0.5em}
		\[
		\texttt{0100\ 0011\ 1001\ 0001\ 1000\ 1001\ 0011\ 0111}
		\]

		---	
	\end{center}
	%ddddddddddddddddddddddddddddddddddddddddddddddddddddddddddddddd
		\begin{center}	
		\colorbox{yellow}{{\textbf{e.} $-6.2265625$}}
		
		\subsection*{Bit de signo}
		
		\[
		\text{El número es negativo} \Rightarrow \text{bit de signo = } \boxed{1}
		\]
		
		\subsection*{Conversión a binario}
		
		
		~~~~~~~~~~~~~~~~~~~~~~~~~\text{Parte entera:} 
		\begin{center}
			\begin{Verbatim}[formatcom=\centering]
				6 / 2 = 3, residuo 0
				3 / 2 = 1, residuo 1
				1 
				Lectura inversa: 110
			\end{Verbatim}
		\end{center}
		
		\text{Parte decimal: Multiplicamos sucesivamente por 2}
		
		\begin{center}
			\begin{Verbatim}
				bit1: 0.22656 * 2 = 0.45312 0 
				bit2: 0.45312 * 2 = 0.90625 0 
				bit3: 0.90625 * 2 = 1.81250 1 
				bit4: 0.81250 * 2 = 1.62500 1 
				bit5: 0.62500 * 2 = 1.25000 1 
				bit6: 0.25000 * 2 = 0.50000 0 
				bit7: 0.50000 * 2 = 1.00000 1 
				bit8: 0.00000 * 2 = 0.00000 0 				
			\end{Verbatim}
		\end{center}
		
		Resultado binario: 
		\[
		-6.2265625_{10} = -110.00111010_2
		\]
		
		\subsection*{Determinación del exponente}
		
		Se debe aplicar el sesgo de 127 moviendo 2 lugares a la izquierda hasta el primer 1
		\vspace{-0.5em}
		\[
		\text{Exponente real} = 2 \quad \Rightarrow \quad \text{Exponente IEEE} = 2 + 127 = 129
		\]
		\vspace{-0.5em}
		\[
		114_{10} = 10000001_2
		\]
		
		\subsection*{Mantisa (23 bits)}
		
		Se toma lo que sigue después del primer 1 en la forma normalizada:
		\vspace{-0.5em}
		\[
		\text{Mantisa: } \underline{1}.0100111010... \Rightarrow 01001110100000000000000
		\]
		
		\subsection*{Resultado IEEE 754 (formato 32 bits)}
		
		\[
		\text{IEEE 754:} \quad 
		\boxed{1} \quad \boxed{10000001} \quad \boxed{01001110100000000000000}
		\]
		
		Separado en grupos de 4 bits:
		\vspace{-0.5em}
		\[
		\texttt{1100\ 0000\ 1100\ 0111\ 0100\ 0000\ 0000\ 0000}
		\]

		---	
	\end{center}		
%eeeeeeeeeeeeeeeeeeeeeeeeeeeeeeeeeeeeeeeeeeeeeeeeeeeeeeeeeeeeeeeeeeeeeeeeeeeeeeeeeeeeeeeeee
		\begin{center}	
			\colorbox{yellow}{{\textbf{e.} $14$}}
			
			\subsection*{Bit de signo}
			
			\[
			\text{El número es positivo} \Rightarrow \text{bit de signo = } \boxed{0}
			\]
			
			\subsection*{Conversión a binario}
			~~~~~~~\text{Parte entera: Dividimos sucesivamente por 2}
			\begin{center}
				\begin{Verbatim}[formatcom=\centering]
					14 / 2 = 7, residuo 0
					 7 / 2 = 3, residuo 1
					 3 / 2 = 1, residuo 1
					 1
					Lectura inversa: 1110
				\end{Verbatim}
			\end{center}
			
			
			Resultado binario: 
			\[
			14_{10} = 1110_2
			\]
			
			\subsection*{Determinación del exponente}
			
			Se debe aplicar el sesgo de 127 moviendo 3 lugares a la izquierda hasta el primer 1
			\vspace{-0.5em}
			\[
			\text{Exponente real} = 3 \quad \Rightarrow \quad \text{Exponente IEEE} = 3 + 127 = 130
			\]
			\vspace{-0.5em}
			\[
			130_{10} = 10000010_2
			\]
			
			\subsection*{Mantisa (23 bits)}
			
			Se toma lo que sigue después del primer 1 en la forma normalizada:
			\vspace{-0.5em}
			\[
			\text{Mantisa: } \underline{1}.11000000000000000000000
			\]
			
			\subsection*{Resultado IEEE 754 (formato 32 bits)}
			
			\[
			\text{IEEE 754:} \quad 
			\boxed{0} \quad \boxed{10000010} \quad \boxed{11000000000000000000000}
			\]
			
			Separado en grupos de 4 bits:
			\vspace{-0.5em}
			\[
			\texttt{0100\ 0001\ 0110\ 0000\ 0000\ 0000\ 0000\ 0000}
			\]
			
			---	
		\end{center}
	%fffffffffffffffffffffffffffffffffffffffffffffffffffffffffffffffffffffff
	\begin{center}	
		\colorbox{yellow}{{\textbf{f.} $3.5$}}
		
		\subsection*{Bit de signo}
		
		\[
		\text{El número es positivo} \Rightarrow \text{bit de signo = } \boxed{0}
		\]
		
		\subsection*{Conversión a binario}
		~~~~~~~\text{Parte entera: Dividimos sucesivamente por 2}
		\begin{center}
			\begin{Verbatim}[formatcom=\centering]
				3  / 2 = 1,  residuo 1
				1  / 2 = 0,  residuo 1
				
				Lectura inversa: 11
			\end{Verbatim}
		\end{center}
		
		
		~~~~~~~~~\text{Parte decimal: Multiplicamos sucesivamente por 2}
		
		\begin{center}
			\begin{Verbatim}
				0.5 * 2 = 1.00000 → 1 
				0.0 * 2 = 0.00000 → 0 
			\end{Verbatim}
		\end{center}
		
		Resultado binario: 
		\[
		3.5_{10} \approx 11.10
		\]
		
		\subsection*{Determinación del exponente}
		
		Se debe aplicar el sesgo de 127 moviendo 1 lugar a la izquierda hasta el primer 1
		\vspace{-0.5em}
		\[
		\text{Exponente real} = 1 \quad \Rightarrow \quad \text{Exponente IEEE} = 1 + 127 = 128
		\]
		\vspace{-0.5em}
		\[
		128_{10} = 10000000_2
		\]
		
		\subsection*{Mantisa (23 bits)}
		
		Se toma lo que sigue después del primer 1 en la forma normalizada:
		\vspace{-0.5em}
		\[
		\text{Mantisa: } \underline{1}.1100000000... \Rightarrow 11000000000000000000000
		\]
		
		\subsection*{Resultado IEEE 754 (formato 32 bits)}
		
		\[
		\text{IEEE 754:} \quad 
		\boxed{0} \quad \boxed{10000000} \quad \boxed{11000000000000000000000}
		\]
		
		Separado en grupos de 4 bits:
		\vspace{-0.5em}
		\[
		\texttt{0100\ 0000\ 0110\ 0000\ 0000\ 0000\ 0000\ 0000}
		\]
		
		---	
	\end{center}
%gggggggggggggggggggggggggggggggggggggggggggggggggggggggggggggggggggggggggggggggg
			\begin{center}	
			\colorbox{yellow}{{\textbf{g.} $-12.5$}}
			
			\subsection*{Bit de signo}
			
			\[
			\text{El número es negativo} \Rightarrow \text{bit de signo = } \boxed{1}
			\]
			
			\subsection*{Conversión a binario}
			
			
			~~~~~~~~~~~~~~~~~~~~~~~~~\text{Parte entera:} 
			\begin{center}
				\begin{Verbatim}[formatcom=\centering]
					12 / 2 = 6, residuo 0
					 6 / 2 = 3, residuo 0
					 3 / 2 = 1, residuo 1
					 1 
					Lectura inversa: 1100
				\end{Verbatim}
			\end{center}
			
			\text{Parte decimal: Multiplicamos sucesivamente por 2}
			
			\begin{center}
				\begin{Verbatim}
					0.5 * 2 = 1.00000 → 1 
					0.0 * 2 = 0.00000 → 0 
				\end{Verbatim}
			\end{center}
			
			Resultado binario: 
			\[
			-12.5_{10} = -1100.10_2
			\]
			
			\subsection*{Determinación del exponente}
			
			Se debe aplicar el sesgo de 127 moviendo 3 lugares a la izquierda hasta el primer 1
			\vspace{-0.5em}
			\[
			\text{Exponente real} = 3 \quad \Rightarrow \quad \text{Exponente IEEE} = 3 + 127 = 130
			\]
			\vspace{-0.5em}
			\[
			130_{10} = 10000010_2
			\]
			
			\subsection*{Mantisa (23 bits)}
			
			Se toma lo que sigue después del primer 1 en la forma normalizada:
			\vspace{-0.5em}
			\[
			\text{Mantisa: } \underline{1}.10010000000... \Rightarrow 10010000000000000000000
			\]
			
			\subsection*{Resultado IEEE 754 (formato 32 bits)}
			
			\[
			\text{IEEE 754:} \quad 
			\boxed{1} \quad \boxed{10000010} \quad \boxed{10010000000000000000000}
			\]

			Separado en grupos de 4 bits:
			\vspace{-0.5em}
			\[
			\texttt{1100\ 0001\ 0100\ 1000\ 0000\ 0000\ 0000\ 0000}
			\]
			
			---	
		\end{center}
	%hhhhhhhhhhhhhhhhhhhhhhhhhhhhhhhhhhhhhhhhhhhhhhhhhhhhhhhhhhhhhhhhhhhhhhhhhhhhhhhhhhhhhhhh
	\begin{center}	
		\colorbox{yellow}{{\textbf{h.} $10.25$}}
		
		\subsection*{Bit de signo}
		
		\[
		\text{El número es positivo} \Rightarrow \text{bit de signo = } \boxed{0}
		\]
		
		\subsection*{Conversión a binario}
		~~~~~~~\text{Parte entera: Dividimos sucesivamente por 2}
		\begin{center}
			\begin{Verbatim}[formatcom=\centering]
				10 / 2 = 5, residuo 0
				 5 / 2 = 2, residuo 1
				 2 / 2 = 1, residuo 0
				 1
				Lectura inversa: 1010
			\end{Verbatim}
		\end{center}
		
		
		~~~~~~~~~\text{Parte decimal: Multiplicamos sucesivamente por 2}
		
		\begin{center}
			\begin{Verbatim}
				0.25 * 2 = 0.50000 → 0 
				0.50 * 2 = 1.00000 → 1
				0.00 * 2 = 0.00000 → 0
			\end{Verbatim}
		\end{center}
		
		Resultado binario: 
		\[
		3.5_{10} \approx 1010.010
		\]
		
		\subsection*{Determinación del exponente}
		
		Se debe aplicar el sesgo de 127 moviendo 3 lugares a la izquierda hasta el primer 1
		\vspace{-0.5em}
		\[
		\text{Exponente real} = 3 \quad \Rightarrow \quad \text{Exponente IEEE} = 3 + 127 = 130
		\]
		\vspace{-0.5em}
		\[
		130_{10} = 10000010_2
		\]
		
		\subsection*{Mantisa (23 bits)}
		
		Se toma lo que sigue después del primer 1 en la forma normalizada:
		\vspace{-0.5em}
		\[
		\text{Mantisa: } \underline{1}.010010000000... \Rightarrow 01001000000000000000000
		\]
		
		\subsection*{Resultado IEEE 754 (formato 32 bits)}
		
		\[
		\text{IEEE 754:} \quad 
		\boxed{0} \quad \boxed{10000010} \quad \boxed{01001000000000000000000}
		\]
		
		Separado en grupos de 4 bits:
		\vspace{-0.5em}
		\[
		\texttt{0100\ 0001\ 0010\ 0100\ 0000\ 0000\ 0000\ 0000}
		\]

		---	
	\end{center}
	%iiiiiiiiiiiiiiiiiiiiiiiiiiiiiiiiiiiiiiiiiiiiiiiiiiiiiiiiiiiiiiiiiiiiiiiiiiiiiiiiiiiiiiiii
		\begin{center}	
		\colorbox{yellow}{{\textbf{i.} $-6.75$}}
		
		\subsection*{Bit de signo}
		
		\[
		\text{El número es negativo} \Rightarrow \text{bit de signo = } \boxed{1}
		\]
		
		\subsection*{Conversión a binario}
		
		
		~~~~~~~~~~~~~~~~~~~~~~~~~\text{Parte entera:} 
		\begin{center}
			\begin{Verbatim}[formatcom=\centering]
				6 / 2 = 3, residuo 0
				3 / 2 = 1, residuo 1
				1 
				Lectura inversa: 110
			\end{Verbatim}
		\end{center}
		
		\text{Parte decimal: Multiplicamos sucesivamente por 2}
		
		\begin{center}
			\begin{Verbatim}
				0.75 * 2 = 1.50000 → 1 
				0.50 * 2 = 1.00000 → 1
				0.00 * 2 = 0.00000 → 0
			\end{Verbatim}
		\end{center}
		
		Resultado binario: 
		\[
		-6.75_{10} = -110.110_2
		\]
		
		\subsection*{Determinación del exponente}
		
		Se debe aplicar el sesgo de 127 moviendo 2 lugares a la izquierda hasta el primer 1
		\vspace{-0.5em}
		\[
		\text{Exponente real} = 2 \quad \Rightarrow \quad \text{Exponente IEEE} = 3 + 127 = 129
		\]
		\vspace{-0.5em}
		\[
		129_{10} = 10000001_2
		\]
		
		\subsection*{Mantisa (23 bits)}
		
		Se toma lo que sigue después del primer 1 en la forma normalizada:
		\vspace{-0.5em}
		\[
		\text{Mantisa: } \underline{1}.10110000000... \Rightarrow 10110000000000000000000
		\]
		
		\subsection*{Resultado IEEE 754 (formato 32 bits)}
		
		\[
		\text{IEEE 754:} \quad 
		\boxed{1} \quad \boxed{10000001} \quad \boxed{10110000000000000000000}
		\]
		
		Separado en grupos de 4 bits:
		\vspace{-0.5em}
		\[
		\texttt{1100\ 0000\ 1101\ 1000\ 0000\ 0000\ 0000\ 0000}
		\]

		---	
	\end{center}

	2.Obtener el numero en decimal que esta representado en IEEE 754 de 32 bits:
	\begin{enumerate}
		\item 1100 0001 1001 1110 0000 0000 0000 0000
		\item 0100 0000 0010 1001 0000 0000 0000 0000
		\item 1100 0010 0001 0000 0000 0000 0000 0000
		\item 1100 0000 1110 0010 0000 0000 0000 0000
		\item 0100 0001 1011 1010 0000 0000 0000 0000
		\item 1001 1010 1001 0101 0000 0000 0000 0000
		\item 0100 0010 0101 0101 0000 0000 0000 0000
		\item 1100 1000 0000 0011 0000 0000 0000 0000
		\item 0100 0111 1111 0100 0000 0000 0000 0000
	\end{enumerate}
	\vspace{1em}
	\begin{center}
	\colorbox{yellow}{\textbf{a.} 1100 0001 1001 1110 0000 0000 0000 0000}

	\subsection*{Separar campos}
	
	\begin{itemize}
		\item \textbf{Bit de signo:} 1 (número negativo)
		\item \textbf{Exponente:} $10000011_2$ → $131_{10}$
		\item \textbf{Mantisa:} $00111100000000000000000_2$ 
		\[
		0\times2^{-1} + 0\times2^{-2} + 1\times2^{-3} + 1\times2^{-4} + 1\times2^{-5} + 1\times2^{-6} = 0.234375_{10}
		\]
	\end{itemize}
	
	\subsection*{Calcular exponente real}
	
	\[
	\text{Exponente} = 131 - 127 = 4
	\]
	

	
	\subsection*{Reconstrucción del número binario}	
	\[
	(-1)^{\text{signo}} \times 2^{\text{exponente}} \times 1.\text{mantisa}
	\]
	\[
	(-1)^1 \times 2^4 \times 1.234375 = -19.75
	\]
	\vspace{1em}
	\[
	\boxed{-19.75}
	\]
	
	---
	\end{center}

		\begin{center}
		\colorbox{yellow}{\textbf{b.} 0100 0000 0010 1001 0000 0000 0000 0000}
		
		\subsection*{Separar campos}
		
		\begin{itemize}
			\item \textbf{Bit de signo:} 0 (número positivo)
			\item \textbf{Exponente:} $10000000_2$ → $128_{10}$
			\item \textbf{Mantisa:} $01010010000000000000000_2$ 
			\[
			0\times2^{-1} + 1\times2^{-2} + 0\times2^{-3} + 1\times2^{-4} + 0\times2^{-5} + 0\times2^{-6} + 1\times2^{-7} = 0.3203125_{10}
			\]
		\end{itemize}
		
		\subsection*{Calcular exponente real}
		
		\[
		\text{Exponente} = 128 - 127 = 1
		\]
		
		
		
		\subsection*{Reconstrucción del número binario}	
		\[
		(-1)^{\text{signo}} \times 2^{\text{exponente}} \times 1.\text{mantisa}
		\]
		\[
		(-1)^0 \times 2^1 \times 1.3203125 = 2.640625
		\]
		\vspace{1em}
		\[
		\boxed{2.640625}
		\]
		
		---
	\end{center}

	\begin{center}
		\colorbox{yellow}{\textbf{c.} 1100 0010 0001 0000 0000 0000 0000 0000}
		
		\subsection*{Separar campos}
		
		\begin{itemize}
			\item \textbf{Bit de signo:} 1 (número negativo)
			\item \textbf{Exponente:} $10000100_2$ → $132_{10}$
			\item \textbf{Mantisa:} $00100000000000000000000_2$ 
			\[
			0\times2^{-1} + 0\times2^{-2} + 1\times2^{-3} = 0.125_{10}
			\]
		\end{itemize}
		
		\subsection*{Calcular exponente real}
		
		\[
		\text{Exponente} = 132 - 127 = 5
		\]
		
		
		
		\subsection*{Reconstrucción del número binario}	
		\[
		(-1)^{\text{signo}} \times 2^{\text{exponente}} \times 1.\text{mantisa}
		\]
		\[
		(-1)^1 \times 2^5 \times 1.125 = -36
		\]
		\vspace{1em}
		\[
		\boxed{-36}
		\]
		
		---
	\end{center}

	\begin{center}
		\colorbox{yellow}{\textbf{d.} 1100 0000 1110 0010 0000 0000 0000 0000}
		
		\subsection*{Separar campos}
		
		\begin{itemize}
			\item \textbf{Bit de signo:} 1 (número negativo)
			\item \textbf{Exponente:} $10000001_2$ → $129_{10}$
			\item \textbf{Mantisa:} $11000100000000000000000_2$ 
			\[
			1\times2^{-1} + 1\times2^{-2} + 0\times2^{-3} + 0\times2^{-4} + 0\times2^{-5} + 1\times2^{-6} = 0.765625_{10}
			\]
		\end{itemize}
		
		\subsection*{Calcular exponente real}
		
		\[
		\text{Exponente} = 129 - 127 = 2
		\]
		
		
		
		\subsection*{Reconstrucción del número binario}	
		\[
		(-1)^{\text{signo}} \times 2^{\text{exponente}} \times 1.\text{mantisa}
		\]
		\[
		(-1)^1 \times 2^2 \times 1.765625 = -7.0625
		\]
		\vspace{1em}
		\[
		\boxed{-7.0625}
		\]
		
		---
	\end{center}

	

	\begin{center}
	\colorbox{yellow}{\textbf{e.} 0100 0001 1011 1010 0000 0000 0000 0000}
	
	\subsection*{Separar campos}
	
	\begin{itemize}
		\item \textbf{Bit de signo:} 0 (número positivo)
		\item \textbf{Exponente:} $10000011_2$ → $131_{10}$
		\item \textbf{Mantisa:} $01110100000000000000000_2$ 
		\[
		0\times2^{-1} + 1\times2^{-2} + 1\times2^{-3} + 1\times2^{-4} + 0\times2^{-5} + 1\times2^{-6} = 0.453125_{10}
		\]
	\end{itemize}
	
	\subsection*{Calcular exponente real}
	
	\[
	\text{Exponente} = 131 - 127 = 4
	\]
	
	
	
	\subsection*{Reconstrucción del número binario}	
	\[
	(-1)^{\text{signo}} \times 2^{\text{exponente}} \times 1.\text{mantisa}
	\]
	\[
	(-1)^0 \times 2^4 \times 1.453125 = 23.25
	\]
	\vspace{1em}
	\[
	\boxed{23.25}
	\]
	
	---
	\end{center}

	\begin{center}
		\colorbox{yellow}{\textbf{f.} 1001 1010 1001 0101 0000 0000 0000 0000}
		
		\subsection*{Separar campos}
		
		\begin{itemize}
			\item \textbf{Bit de signo:} 1 (número negativo)
			\item \textbf{Exponente:} $00110101_2$ → $53_{10}$
			\item \textbf{Mantisa:} $00101010000000000000000_2$ 
			\[
			0\times2^{-1} + 0\times2^{-2} + 1\times2^{-3} + 0\times2^{-4} + 1\times2^{-5} + 0\times2^{-6} + 1\times2^{-7} = 0.1640625_{10}
			\]
		\end{itemize}
		
		\subsection*{Calcular exponente real}
		
		\[
		\text{Exponente} = 53 - 127 = -74
		\]
		
		
		
		\subsection*{Reconstrucción del número binario}	
		\[
		(-1)^{\text{signo}} \times 2^{\text{exponente}} \times 1.\text{mantisa}
		\]
		\[
		(-1)^1 \times 2^-74 \times 1.1640625 = -6.162495564*10^{-23}
		\]
		\vspace{1em}
		\[
		\boxed{-6.162495564*10^{-23}}
		\]
		
		---
	\end{center}

	\begin{center}
		\colorbox{yellow}{\textbf{g.} 0100 0010 0101 0101 0000 0000 0000 0000}
		
		\subsection*{Separar campos}
		
		\begin{itemize}
			\item \textbf{Bit de signo:} 0 (número positivo)
			\item \textbf{Exponente:} $10000100_2$ → $132_{10}$
			\item \textbf{Mantisa:} $10101010000000000000000_2$ 
			\[
			1\times2^{-1} + 0\times2^{-2} + 1\times2^{-3} + 0\times2^{-4} + 1\times2^{-5} + 0\times2^{-6} + 1\times2^{-7} = 0.6640625_{10}
			\]
		\end{itemize}
		
		\subsection*{Calcular exponente real}
		
		\[
		\text{Exponente} = 132 - 127 = 5
		\]
		
		
		
		\subsection*{Reconstrucción del número binario}	
		\[
		(-1)^{\text{signo}} \times 2^{\text{exponente}} \times 1.\text{mantisa}
		\]
		\[
		(-1)^0 \times 2^5 \times 1.6640625 = 53.25
		\]
		\vspace{1em}
		\[
		\boxed{53.25}
		\]
		
		---
	\end{center}

	\begin{center}
		\colorbox{yellow}{\textbf{h.} 1100 1000 0000 0011 0000 0000 0000 0000}
		
		\subsection*{Separar campos}
		
		\begin{itemize}
			\item \textbf{Bit de signo:} 1 (número negativo)
			\item \textbf{Exponente:} $10010000_2$ → $144_{10}$
			\item \textbf{Mantisa:} $00000110000000000000000_2$ 
			\[
			0\times2^{-1} + 0\times2^{-2} + 0\times2^{-3} + 0\times2^{-4} + 0\times2^{-5} + 1\times2^{-6} + 1\times2^{-7} = 0.0234375_{10}
			\]
		\end{itemize}
		
		\subsection*{Calcular exponente real}
		
		\[
		\text{Exponente} = 144 - 127 = 17
		\]
		
		
		
		\subsection*{Reconstrucción del número binario}	
		\[
		(-1)^{\text{signo}} \times 2^{\text{exponente}} \times 1.\text{mantisa}
		\]
		\[
		(-1)^1 \times 2^17 \times 1.0234375 = -134144
		\]
		\vspace{1em}
		\[
		\boxed{-134144}
		\]
		
		---
	\end{center}

	\begin{center}
		\colorbox{yellow}{\textbf{i.} 0100 0111 1111 0100 0000 0000 0000 0000}
		
		\subsection*{Separar campos}
		
		\begin{itemize}
			\item \textbf{Bit de signo:} 1 (número negativo)
			\item \textbf{Exponente:} $10001111_2$ → $143_{10}$
			\item \textbf{Mantisa:} $11101000000000000000000_2$ 
			\[
			1\times2^{-1} + 1\times2^{-2} + 1\times2^{-3} + 0\times2^{-4} + 1\times2^{-5} = 0.90625_{10}
			\]
		\end{itemize}
		
		\subsection*{Calcular exponente real}
		
		\[
		\text{Exponente} = 143 - 127 = 16
		\]
		
		
		
		\subsection*{Reconstrucción del número binario}	
		\[
		(-1)^{\text{signo}} \times 2^{\text{exponente}} \times 1.\text{mantisa}
		\]
		\[
		(-1)^1 \times 2^16 \times 1.90625 = -124928
		\]
		\vspace{1em}
		\[
		\boxed{-124928}
		\]
		
		---
	\end{center}
	
	3.Indicar el valor decimal de los siguientes números hexadecimales que siguen
	el formato de coma flotante IEEE 754.
	\begin{enumerate}
		\item FF800000
		\item 7F804000
		\item C7B00000
		\item 0A180000
		\item 40E00000
		\item BF400000
		\item 804B0000
		\item 42378000
		\item B7890000
	\end{enumerate}	



	
	\vspace{1em}
	%AAAAAAAAAAAAAAAAAAAAAA
	\begin{center}
	\colorbox{yellow}{\textbf{a.} FF800000}
	\subsection*{Conversión a binario (32 bits)}
	Cada dígito hexadecimal equivale a 4 bits:
	
	\[
		\mathtt{
		\underset{\text{F}}{\boxed{\mathtt{1111}}}
		\underset{\text{F}}{\boxed{\mathtt{1111}}}
		\underset{\text{8}}{\boxed{\mathtt{1000}}}
		\underset{\text{0}}{\boxed{\mathtt{0000}}}
		\underset{\text{0}}{\boxed{\mathtt{0000}}}
		\underset{\text{0}}{\boxed{\mathtt{0000}}}
		\underset{\text{0}}{\boxed{\mathtt{0000}}}
		\underset{\text{0}}{\boxed{\mathtt{0000}}}
		}	
	\]
	
	\subsection*{Separar campos}
	
	\begin{itemize}
		\item \textbf{Bit de signo:} 1 (número negativo)
		\item \textbf{Exponente:} $11111111_2$ → $255_{10}$
		\item \textbf{Mantisa:} $00000000000000000000000_2$ 
	\end{itemize}
	
	\subsection*{Interpretación del número}
	
	- Exponente = 255 → valor reservado en IEEE 754\newline
	- Si la mantisa != 0 → representa un \textbf{NaN (Not a Number)}\newline
	- Si la mantisa = 0 → representa un \textbf{infinito positivo/negativo}
	
	\[
	\Rightarrow \text{Este número representa: } \boxed{-\infty}
	\]
	
	---
	\end{center}
	%BBBBBBBBBBBBBBBBBBBBBBBBBBBBBBBBBBBBBBBBBBBBBBBBB
	\begin{center}
		\colorbox{yellow}{\textbf{b.} 7F804000}
		\subsection*{Conversión a binario (32 bits)}
		Cada dígito hexadecimal equivale a 4 bits:
		
		\[
		\mathtt{
			\underset{\text{7}}{\boxed{\mathtt{0111}}}
			\underset{\text{F}}{\boxed{\mathtt{1111}}}
			\underset{\text{8}}{\boxed{\mathtt{1000}}}
			\underset{\text{0}}{\boxed{\mathtt{0000}}}
			\underset{\text{4}}{\boxed{\mathtt{0100}}}
			\underset{\text{0}}{\boxed{\mathtt{0000}}}
			\underset{\text{0}}{\boxed{\mathtt{0000}}}
			\underset{\text{0}}{\boxed{\mathtt{0000}}}
		}	
		\]
		
		\subsection*{Separar campos}
		
		\begin{itemize}
			\item \textbf{Bit de signo:} 1 (número negativo)
			\item \textbf{Exponente:} $11111111_2$ → $255_{10}$
			\item \textbf{Mantisa:} $00000000100000000000000_2$ 
			\[
			0\times2^{-1} + 0\times2^{-2} + 0\times2^{-3} + 0\times2^{-4} + 0\times2^{-5} + 0\times2^{-6} + 0\times2^{-7} + 0\times2^{-8} + 0\times2^{-9}= 0.001953125_{10}
			\]
		\end{itemize}
		
		\subsection*{Interpretación del número}
		
		- Exponente = 255 → valor reservado en IEEE 754\newline
		- Si la mantisa != 0 → representa un \textbf{NaN (Not a Number)}\newline
		- Si la mantisa = 0 → representa un \textbf{infinito positivo/negativo}
		
		\[
		\Rightarrow \text{Este número representa: } \boxed{NaN}
		\]
		
		---
	\end{center}
	%CCCCCCCCCCCCCCCCCCCCCCCCCCCCCCCCCCCCCCCCCCCCCCCCC
	\begin{center}
		\colorbox{yellow}{\textbf{c.} C7B00000}
		\subsection*{Conversión a binario (32 bits)}
		Cada dígito hexadecimal equivale a 4 bits:
		
		\[
		\mathtt{
			\underset{\text{C}}{\boxed{\mathtt{1100}}}
			\underset{\text{7}}{\boxed{\mathtt{0111}}}
			\underset{\text{B}}{\boxed{\mathtt{1011}}}
			\underset{\text{0}}{\boxed{\mathtt{0000}}}
			\underset{\text{0}}{\boxed{\mathtt{0000}}}
			\underset{\text{0}}{\boxed{\mathtt{0000}}}
			\underset{\text{0}}{\boxed{\mathtt{0000}}}
			\underset{\text{0}}{\boxed{\mathtt{0000}}}
		}	
		\]
		
		\subsection*{Separar campos}
		
		\begin{itemize}
			\item \textbf{Bit de signo:} 1 (número negativo)
			\item \textbf{Exponente:} $10001111_2$ → $15_{10}$
			\item \textbf{Mantisa:} $01100000000000000000000_2$ 
			\[
			0\times2^{-1} + 1\times2^{-2} + 1\times2^{-3} = 0.375_{10}
			\]
		\end{itemize}
		
		\subsection*{Calcular exponente real}
		
		\[
		\text{Exponente} = 143 - 127 = 16
		\]
		
		
		
		\subsection*{Reconstrucción del número binario}	
		\[
		(-1)^{\text{signo}} \times 2^{\text{exponente}} \times 1.\text{mantisa}
		\]
		\[
		(-1)^1 \times 2^16 \times 1.375 = -90112
		\]
		\vspace{1em}
		\[
		\boxed{-90112}
		\]
		
		---
	\end{center}
	%DDDDDDDDDDDDDDDDDDDDDDDDDDDDDDDDDDDDDDDDDDDDDDDDD
	\begin{center}
		\colorbox{yellow}{\textbf{d.} 0A180000}
		\subsection*{Conversión a binario (32 bits)}
		Cada dígito hexadecimal equivale a 4 bits:
		
		\[
		\mathtt{
			\underset{\text{0}}{\boxed{\mathtt{0000}}}
			\underset{\text{A}}{\boxed{\mathtt{1010}}}
			\underset{\text{1}}{\boxed{\mathtt{0001}}}
			\underset{\text{8}}{\boxed{\mathtt{1000}}}
			\underset{\text{0}}{\boxed{\mathtt{0000}}}
			\underset{\text{0}}{\boxed{\mathtt{0000}}}
			\underset{\text{0}}{\boxed{\mathtt{0000}}}
			\underset{\text{0}}{\boxed{\mathtt{0000}}}
		}	
		\]
		
		\subsection*{Separar campos}
		
		\begin{itemize}
			\item \textbf{Bit de signo:} 0 (número positivo)
			\item \textbf{Exponente:} $00010100_2$ → $20_{10}$
			\item \textbf{Mantisa:} $00110000000000000000000_2$ 
			\[
			0\times2^{-1} + 0\times2^{-2} + 1\times2^{-3} + 1\times2^{-4} = 0.1875_{10}
			\]
		\end{itemize}
		
		\subsection*{Calcular exponente real}
		
		\[
		\text{Exponente} = 20 - 127 = -107
		\]
		
		
		
		\subsection*{Reconstrucción del número binario}	
		\[
		(-1)^{\text{signo}} \times 2^{\text{exponente}} \times 1.\text{mantisa}
		\]
		\[
		(-1)^0 \times 2^-107 \times 1.1875 = 7.318533789*10^{-33}
		\]
		\vspace{1em}
		\[
		\boxed{7.318533789*10^{-33}}
		\]
		
		---
	\end{center}
	%EEEEEEEEEEEEEEEEEEEEEEEEEEEEEEEEEEEEEEEEEEEEEEEEEEEEEEEEEEEEE
	\begin{center}
		\colorbox{yellow}{\textbf{e.} 40E00000}
		\subsection*{Conversión a binario (32 bits)}
		Cada dígito hexadecimal equivale a 4 bits:
		
		\[
		\mathtt{
			\underset{\text{4}}{\boxed{\mathtt{0100}}}
			\underset{\text{0}}{\boxed{\mathtt{0000}}}
			\underset{\text{E}}{\boxed{\mathtt{1110}}}
			\underset{\text{0}}{\boxed{\mathtt{0000}}}
			\underset{\text{0}}{\boxed{\mathtt{0000}}}
			\underset{\text{0}}{\boxed{\mathtt{0000}}}
			\underset{\text{0}}{\boxed{\mathtt{0000}}}
			\underset{\text{0}}{\boxed{\mathtt{0000}}}
		}	
		\]
		
		\subsection*{Separar campos}
		
		\begin{itemize}
			\item \textbf{Bit de signo:} 0 (número positivo)
			\item \textbf{Exponente:} $10000001_2$ → $129_{10}$
			\item \textbf{Mantisa:} $11000000000000000000000_2$ 
			\[
			1\times2^{-1} + 1\times2^{-2} = 0.75_{10}
			\]
		\end{itemize}
		
		\subsection*{Calcular exponente real}
		
		\[
		\text{Exponente} = 129 - 127 = 2
		\]
		
		
		
		\subsection*{Reconstrucción del número binario}	
		\[
		(-1)^{\text{signo}} \times 2^{\text{exponente}} \times 1.\text{mantisa}
		\]
		\[
		(-1)^0 \times 2^2 \times 1.75 = 7
		\]
		\vspace{1em}
		\[
		\boxed{7}
		\]
		
		---
	\end{center}
	%FFFFFFFFFFFFFFFFFFFFFFFFFFFFFFFFFFFFFFFFFFFFFFFFF
	\begin{center}
		\colorbox{yellow}{\textbf{f.} BF400000}
		\subsection*{Conversión a binario (32 bits)}
		Cada dígito hexadecimal equivale a 4 bits:
		
		\[
		\mathtt{
			\underset{\text{B}}{\boxed{\mathtt{0111}}}
			\underset{\text{F}}{\boxed{\mathtt{1111}}}
			\underset{\text{4}}{\boxed{\mathtt{0100}}}
			\underset{\text{0}}{\boxed{\mathtt{0000}}}
			\underset{\text{0}}{\boxed{\mathtt{0000}}}
			\underset{\text{0}}{\boxed{\mathtt{0000}}}
			\underset{\text{0}}{\boxed{\mathtt{0000}}}
			\underset{\text{0}}{\boxed{\mathtt{0000}}}
		}	
		\]
		
		\subsection*{Separar campos}
		
		\begin{itemize}
			\item \textbf{Bit de signo:} 0 (número positivo)
			\item \textbf{Exponente:} $11111110_2$ → $254_{10}$
			\item \textbf{Mantisa:} $10000000000000000000000_2$ 
			\[
			1\times2^{-1}  = 0.5_{10}
			\]
		\end{itemize}
		
		\subsection*{Calcular exponente real}
		
		\[
		\text{Exponente} = 254 - 127 = 127
		\]
		
		
		
		\subsection*{Reconstrucción del número binario}	
		\[
		(-1)^{\text{signo}} \times 2^{\text{exponente}} \times 1.\text{mantisa}
		\]
		\[
		(-1)^0 \times 2^127 \times 1.5 = 2.552117752*10^{38}
		\]
		\vspace{1em}
		\[
		\boxed{2.552117752*10^{38}}
		\]
		
		---
	\end{center}
	%GGGGGGGGGGGGGGGGGGGGGGGGGGGGGGGGGGGGGGGGGGGGGGGGGGGGGGGG
	\begin{center}
		\colorbox{yellow}{\textbf{g.} 804B0000}
		\subsection*{Conversión a binario (32 bits)}
		Cada dígito hexadecimal equivale a 4 bits:
		
		\[
		\mathtt{
			\underset{\text{8}}{\boxed{\mathtt{1000}}}
			\underset{\text{0}}{\boxed{\mathtt{0000}}}
			\underset{\text{4}}{\boxed{\mathtt{0100}}}
			\underset{\text{B}}{\boxed{\mathtt{0111}}}
			\underset{\text{0}}{\boxed{\mathtt{0000}}}
			\underset{\text{0}}{\boxed{\mathtt{0000}}}
			\underset{\text{0}}{\boxed{\mathtt{0000}}}
			\underset{\text{0}}{\boxed{\mathtt{0000}}}
		}	
		\]
		
		\subsection*{Separar campos}
		
		\begin{itemize}
			\item \textbf{Bit de signo:} 1 (número negativo)
			\item \textbf{Exponente:} $00000000_2$ → como todos los bits son cero, se trata de un número subnormal, y por lo tanto el exponente real es -126 (no se aplica la resta con el sesgo 127 en este caso).
			\item \textbf{Mantisa:} $10001110000000000000000_2$ 
			\[
			1\times2^{-1} + 0\times2^{-2} + 0\times2^{-3} + 0\times2^{-4} + 1\times2^{-5} + 1\times2^{-6} + 1\times2^{-7} = 0.5546875_{10}
			\]
		\end{itemize}
		
		\subsection*{Calcular exponente real}
		
		\[
		\text{Exponente} = -126
		\]
		
		
		
		\subsection*{Reconstrucción del número binario}
		La mantisa se interpreta como 0.fracción binaria, sin anteponer un 1 cuando el exponente es 0 (caso subnormal).	
		\[
		(-1)^{\text{signo}} \times 2^{\text{exponente}} \times 0.\text{mantisa}
		\]
		\[
		(-1)^1 \times 2^-126 \times 0.5546875 = -6.520320227*10^{-39}
		\]
		\vspace{1em}
		\[
		\boxed{-6.520320227*10^{-39}}
		\]
		
		---
	\end{center}
	%HHHHHHHHHHHHHHHHHHHHHHHHHHHHHHHHHHHHHHHHHHHHHHHHHHHh
	\begin{center}
		\colorbox{yellow}{\textbf{h.} 42378000}
		\subsection*{Conversión a binario (32 bits)}
		Cada dígito hexadecimal equivale a 4 bits:
		
		\[
		\mathtt{
			\underset{\text{4}}{\boxed{\mathtt{0100}}}
			\underset{\text{2}}{\boxed{\mathtt{0010}}}
			\underset{\text{3}}{\boxed{\mathtt{0011}}}
			\underset{\text{7}}{\boxed{\mathtt{0111}}}
			\underset{\text{8}}{\boxed{\mathtt{1000}}}
			\underset{\text{0}}{\boxed{\mathtt{0000}}}
			\underset{\text{0}}{\boxed{\mathtt{0000}}}
			\underset{\text{0}}{\boxed{\mathtt{0000}}}
		}	
		\]
		
		\subsection*{Separar campos}
		
		\begin{itemize}
			\item \textbf{Bit de signo:} 0 (número positivo)
			\item \textbf{Exponente:} $10000100_2$ → $132_{10}$
			\item \textbf{Mantisa:} $01101111000000000000000_2$ 
			\[
			0\times2^{-1} + 1\times2^{-2} + 1\times2^{-3} + 0\times2^{-4} + 1\times2^{-5} + 1\times2^{-6} + 1\times2^{-7} + 1\times2^{-8}= 0.43359375_{10}
			\]
		\end{itemize}
		
		\subsection*{Calcular exponente real}
		
		\[
		\text{Exponente} = 132 -127 = 5
		\]
		
		
		
		\subsection*{Reconstrucción del número binario}	
		\[
		(-1)^{\text{signo}} \times 2^{\text{exponente}} \times 1.\text{mantisa}
		\]
		\[
		(-1)^0 \times 2^5 \times 1.43359375 = 45.875
		\]
		\vspace{1em}
		\[
		\boxed{45.875}
		\]
		
		---
	\end{center}
	%IIIIIIIIIIIIIIIIIIIIIIIIIIIIIIIIIIIIIIIIIIIIIIIII
	\begin{center}
		\colorbox{yellow}{\textbf{i.} B7890000}
		\subsection*{Conversión a binario (32 bits)}
		Cada dígito hexadecimal equivale a 4 bits:
		
		\[
		\mathtt{
			\underset{\text{B}}{\boxed{\mathtt{1011}}}
			\underset{\text{7}}{\boxed{\mathtt{0111}}}
			\underset{\text{8}}{\boxed{\mathtt{1000}}}
			\underset{\text{9}}{\boxed{\mathtt{1001}}}
			\underset{\text{0}}{\boxed{\mathtt{0000}}}
			\underset{\text{0}}{\boxed{\mathtt{0000}}}
			\underset{\text{0}}{\boxed{\mathtt{0000}}}
			\underset{\text{0}}{\boxed{\mathtt{0000}}}
		}	
		\]
		
		\subsection*{Separar campos}
		
		\begin{itemize}
			\item \textbf{Bit de signo:} 1 (número negativo)
			\item \textbf{Exponente:} $01101111_2$ → $111_{10}$
			\item \textbf{Mantisa:} $00010010000000000000000_2$ 
			\[
			0\times2^{-1} + 0\times2^{-2} + 0\times2^{-3} + 1\times2^{-4} + 0\times2^{-5} + 0\times2^{-6} + 1\times2^{-7} = 0.0703125_{10}
			\]
		\end{itemize}
		
		\subsection*{Calcular exponente real}
		
		\[
		\text{Exponente} = 111 -127 = -16
		\]
		
		
		
		\subsection*{Reconstrucción del número binario}	
		\[
		(-1)^{\text{signo}} \times 2^{\text{exponente}} \times 1.\text{mantisa}
		\]
		\[
		(-1)^0 \times 2^-16 \times 1.0703125 = -0.000016331673
		\]
		\vspace{1em}
		\[
		\boxed{-0.000016331673}
		\]
		
		---
	\end{center}

	4.Convertir los siguientes números binarios a sus equivalentes decimales:
	\begin{enumerate}
		\item 001100 
		\item 000011 
		\item 011100 
		\item 111100 
		\item 101010 
		\item 111111 
		\item 100001 
		\item 111000 
		\item 11110001111 
		\item 11100,011 
		\item 110011,10011 
		\item 1010101010,1
	\end{enumerate}
	
	\begin{center}
	\colorbox{yellow}{{\textbf{a.} $001100$}}
	\subsection*{Calculo mediante el teorema fundamental de la numeración}
	\[
	0\times2^{5} + 0\times2^{4} + 1\times2^{3} + 1\times2^{2} + 0\times2^{1} + 0\times2^{0}
	\]
	\boxed{12_{10}}
	\end{center}
	\begin{center}
		\colorbox{yellow}{{\textbf{b.} $000011$}}
		\subsection*{Calculo mediante el teorema fundamental de la numeración}
		\[
		0\times2^{5} + 0\times2^{4} + 0\times2^{3} + 0\times2^{2} + 1\times2^{1} + 1\times2^{0}
		\]
		\boxed{3_{10}}
	\end{center}
	\begin{center}
		\colorbox{yellow}{{\textbf{c.} $011100$}}
		\subsection*{Calculo mediante el teorema fundamental de la numeración}
		\[
		0\times2^{5} + 1\times2^{4} + 1\times2^{3} + 1\times2^{2} + 0\times2^{1} + 0\times2^{0}
		\]
		\boxed{28_{10}}
	\end{center}
	\begin{center}
		\colorbox{yellow}{{\textbf{d.} $111100$}}
		\subsection*{Calculo mediante el teorema fundamental de la numeración}
		\[
		1\times2^{5} + 1\times2^{4} + 1\times2^{3} + 1\times2^{2} + 0\times2^{1} + 0\times2^{0}
		\]
		\boxed{60_{10}}
	\end{center}
	\begin{center}
		\colorbox{yellow}{{\textbf{e.} $101010$}}
		\subsection*{Calculo mediante el teorema fundamental de la numeración}
		\[
		1\times2^{5} + 0\times2^{4} + 1\times2^{3} + 0\times2^{2} + 1\times2^{1} + 0\times2^{0}
		\]
		\boxed{42_{10}}
	\end{center}
	\begin{center}
		\colorbox{yellow}{{\textbf{f.} $111111$}}
		\subsection*{Calculo mediante el teorema fundamental de la numeración}
		\[
		1\times2^{5} + 1\times2^{4} + 1\times2^{3} + 1\times2^{2} + 1\times2^{1} + 1\times2^{0}
		\]
		\boxed{63_{10}}
	\end{center}
	\begin{center}
		\colorbox{yellow}{{\textbf{g.} $100001$}}
		\subsection*{Calculo mediante el teorema fundamental de la numeración}
		\[
		1\times2^{5} + 0\times2^{4} + 0\times2^{3} + 0\times2^{2} + 0\times2^{1} + 1\times2^{0}
		\]
		\boxed{33_{10}}
	\end{center}
	\begin{center}
		\colorbox{yellow}{{\textbf{h.} $111000$}}
		\subsection*{Calculo mediante el teorema fundamental de la numeración}
		\[
		1\times2^{5} + 1\times2^{4} + 1\times2^{3} + 0\times2^{2} + 0\times2^{1} + 0\times2^{0}
		\]
		\boxed{56_{10}}
	\end{center}
	\begin{center}
		\colorbox{yellow}{{\textbf{i.} $11110001111$}}
		\subsection*{Calculo mediante el teorema fundamental de la numeración}
		\[
		1\times2^{10} + 1\times2^{9} + 1\times2^{8} + 1\times2^{7} + 0\times2^{6} + 0\times2^{5} + 0\times2^{4} + 1\times2^{3} + 1\times2^{2} + 1\times2^{1} + 1\times2^{0}
		\]
		\boxed{1935_{10}}
	\end{center}
	\begin{center}
		\colorbox{yellow}{{\textbf{j.} $11100,011$}}
		\subsection*{Calculo mediante el teorema fundamental de la numeración}
		\[
		1\times2^{4} + 1\times2^{3} + 1\times2^{2} + 0\times2^{1} + 0\times2^{0} + 0\times2^{-1} + 1\times2^{-2} + 1\times2^{-3} 
		\]
		\boxed{28,375_{10}}
	\end{center}
	\begin{center}
		\colorbox{yellow}{{\textbf{k.} $110011,10011$}}
		\subsection*{Calculo mediante el teorema fundamental de la numeración}
		\[
		1\times2^{5} + 1\times2^{4} + 0\times2^{3} + 0\times2^{2} + 1\times2^{1} + 1\times2^{0} + 1\times2^{-1} + 0\times2^{-2} + 0\times2^{-3} + 1\times2^{-4} + 1\times2^{-5}
		\]
		\boxed{51,59375_{10}}
	\end{center}
	\begin{center}
		\colorbox{yellow}{{\textbf{l.} $1010101010,1$}}
	
		\[
		1\times2^{9} + 0\times2^{8} + 1\times2^{7} + 0\times2^{6} + 1\times2^{5} + 0\times2^{4} + 1\times2^{3} + 0\times2^{2} + 1\times2^{1} + 0\times2^{0} + 1\times2^{-1}
		\]
		\boxed{682.5_{10}}
	\end{center}

	5.Convertir los siguientes números decimales a sus equivalentes binarios:
		\begin{enumerate}
			\item 64 
			\item 100 
			\item 111 
			\item 145 
			\item 255 
			\item 500 
			\item 34,75 
			\item 25,25 
			\item 27,1875 
			\item 23,1
		\end{enumerate}
	
		
		\begin{center}
			\colorbox{yellow}{{\textbf{a.} $64$}}
			\begin{Verbatim}[formatcom=\centering]
				64 / 2 = 32, residuo 0
				32 / 2 = 16, residuo 0
				16 / 2 =  8, residuo 0
				 8 / 2 =  4, residuo 0
				 4 / 2 =  2, residuo 0
				 2 / 2 =  1, residuo 0
				 1 ----------------> 1
			\end{Verbatim}
		\hspace{1.7cm}\boxed{\texttt{Lectura inversa: 1000000}}
		\end{center}
		\begin{center}
			\colorbox{yellow}{{\textbf{b.} $100$}}
			\begin{Verbatim}[formatcom=\centering]
				100 / 2 = 50, residuo 0
				 50 / 2 = 25, residuo 0
				 25 / 2 = 12, residuo 1
				 12 / 2 =  6, residuo 0
				  6 / 2 =  3, residuo 0
				  3 / 2 =  1, residuo 1
				  1 ----------------> 1
			\end{Verbatim}
			\hspace{1.7cm}\boxed{\texttt{Lectura inversa: 1100100}}
		\end{center}
		\begin{center}
			\colorbox{yellow}{{\textbf{c.} $111$}}
			\begin{Verbatim}[formatcom=\centering]
				111 / 2 = 55, residuo 1
				 55 / 2 = 27, residuo 1
				 27 / 2 = 13, residuo 1
				 13 / 2 =  6, residuo 1
				  6 / 2 =  3, residuo 0
				  3 / 2 =  1, residuo 1
				  1 ----------------> 1
			\end{Verbatim}
			\hspace{1.7cm}\boxed{\texttt{Lectura inversa: 1101111}}
		\end{center}
		\begin{center}
			\colorbox{yellow}{{\textbf{d.} $145$}}
			\begin{Verbatim}[formatcom=\centering]
				145 / 2 = 72, residuo 1
				 72 / 2 = 36, residuo 0
				 36 / 2 = 18, residuo 0
				 18 / 2 =  9, residuo 0
				  9 / 2 =  4, residuo 1
			   	  4 / 2 =  2, residuo 0
			   	  2 / 2 =  1, residuo 0
				  1 ----------------> 1
			\end{Verbatim}
			\hspace{1.7cm}\boxed{\texttt{Lectura inversa: 10010001}}
		\end{center}
		\begin{center}
			\colorbox{yellow}{{\textbf{e.} $255$}}
			\begin{Verbatim}[formatcom=\centering]
				255 / 2 = 127, residuo 1
				127 / 2 =  63, residuo 1
				 63 / 2 =  31, residuo 1
 				 31 / 2 =  15, residuo 1
				 15 / 2 =   7, residuo 1
				  7 / 2 =   3, residuo 1
				  3 / 2 =   1, residuo 1
				  1 -----------------> 1
			\end{Verbatim}
			\hspace{1.7cm}\boxed{\texttt{Lectura inversa: 11111111}}
		\end{center}
		\begin{center}
			\colorbox{yellow}{{\textbf{f.} $500$}}
			\begin{Verbatim}[formatcom=\centering]
				500 / 2 = 250, residuo 0
				250 / 2 = 125, residuo 0
				125 / 2 =  62, residuo 1
				 62 / 2 =  31, residuo 0
				 31 / 2 =  15, residuo 1
				 15 / 2 =   7, residuo 1
				  7 / 2 =   3, residuo 1
				  3 / 2 =   1, residuo 1
				  1 -----------------> 1
			\end{Verbatim}
			\hspace{1.7cm}\boxed{\texttt{Lectura inversa: 111110100}}
		\end{center}
		\begin{center}
			\colorbox{yellow}{{\textbf{g.} $34,75$}} \\ \vspace{0.3cm}
		\end{center}	
		\vspace{-1em}
	%ACA EMPIEZA EL MINIPAGE
		\begin{minipage}[t]{0.40\textwidth}
		\hspace{2,5cm}\text{Parte entera: 34}
		
\begin{Verbatim}
	34 / 2 = 17, residuo 0
	17 / 2 =  8, residuo 1
         8 / 2 =  4, residuo 0
	 4 / 2 =  2, residuo 0
	 2 / 2 =  1, residuo 0
	 1 ----------------> 1
\end{Verbatim}
		\hspace{1,5cm}\texttt{Lectura inversa: 100010}
		\end{minipage}
	\hfill
	\begin{minipage}[t]{0.40\textwidth}
	\hspace{2cm}\text{Parte decimal: 0,75}
	
\begin{Verbatim}
	0.75 * 2 = 1.5 → 1 
	0.50 * 2 = 1.0 → 1 
	0.00 * 2 = 0.0 → 0 
\end{Verbatim}
			\hspace{1.7cm}{\texttt{Parte decimal: 0,11}}			
	\end{minipage}
\begin{center}
\hspace{2cm}\boxed{\texttt{Número completo: 100010,11}}
\end{center}

\begin{center}
	\colorbox{yellow}{{\textbf{h.} $25,25$}} \\ \vspace{0.3cm}
\end{center}	
\vspace{-1em}
%ACA EMPIEZA EL MINIPAGE
\begin{minipage}[t]{0.40\textwidth}
	\hspace{2,5cm}\text{Parte entera: 25}
	
	\begin{Verbatim}
	25 / 2 = 12, residuo 1
	12 / 2 =  6, residuo 0
	6 / 2 =   3, residuo 0
	3 / 2 =   1, residuo 1
	1 -----------------> 1
	\end{Verbatim}
	\hspace{1,5cm}\texttt{Lectura inversa: 11001}
\end{minipage}
\hfill
\begin{minipage}[t]{0.40\textwidth}
	\hspace{2cm}\text{Parte decimal: 0,25}
	
	\begin{Verbatim}
	0.25 * 2 = 0.5 → 0
	0.50 * 2 = 1.0 → 1
	0.00 * 2 = 0.0 → 0 
	\end{Verbatim}
	\hspace{1.7cm}{\texttt{Parte decimal: 0,01}}			
\end{minipage}
\begin{center}
	\hspace{2cm}\boxed{\texttt{Número completo: 11001,01}}
\end{center}

\begin{center}
	\colorbox{yellow}{{\textbf{i.} $25,25$}} \\ \vspace{0.3cm}
\end{center}	
\vspace{-1em}
%ACA EMPIEZA EL MINIPAGE
\begin{minipage}[t]{0.40\textwidth}
	\hspace{2,5cm}\text{Parte entera: 25}
	
	\begin{Verbatim}
	25 / 2 = 12, residuo 1
	12 / 2 =  6, residuo 0
	6 / 2 =   3, residuo 0
	3 / 2 =   1, residuo 1
	1 -----------------> 1
	\end{Verbatim}
	\hspace{1,5cm}\texttt{Lectura inversa: 11001}
\end{minipage}
\hfill
\begin{minipage}[t]{0.40\textwidth}
	\hspace{2cm}\text{Parte decimal: 0,75}
	
	\begin{Verbatim}
	0.25 * 2 = 0.5 → 0
	0.50 * 2 = 1.0 → 1
	0.00 * 2 = 0.0 → 0 
	\end{Verbatim}
	\hspace{1.7cm}{\texttt{Parte decimal: 0,01}}			
\end{minipage}
\begin{center}
	\hspace{2cm}\boxed{\texttt{Número completo: 11001,01}}
\end{center}

\begin{center}
	\colorbox{yellow}{{\textbf{j.} $23,1$}} \\ \vspace{0.3cm}
\end{center}	
\vspace{-1em}
%ACA EMPIEZA EL MINIPAGE
\begin{minipage}[t]{0.40\textwidth}
	\hspace{2,5cm}\text{Parte entera: 23}
	
	\begin{Verbatim}
	23 / 2 = 11, residuo 1
	11 / 2 =  5, residuo 1
	 5 / 2 =  2, residuo 1
	 2 / 2 =  1, residuo 0
	 1 ----------------> 1
	\end{Verbatim}
	\hspace{1,5cm}\texttt{Lectura inversa: 101111}
\end{minipage}
\hfill
\begin{minipage}[t]{0.40\textwidth}
	\hspace{2cm}\text{Parte decimal: 0,1}
	
	\begin{Verbatim}
	0.1 * 2 = 0.20000 → 0 
	0.2 * 2 = 0.40000 → 0 
	0.4 * 2 = 0.80000 → 0 
	0.8 * 2 = 1.60000 → 1 
	0.6 * 2 = 1.20000 → 1 
	0.2 * 2 = 0.40000 → 0 
	0.4 * 2 = 0.80000 → 0 
	0.8 * 2 = 1.60000 → 1 
	0.6 * 2 = 1.20000 → 1 
	0.2 * 2 = 0.40000 → 0 
	0.4 * 2 = 0.80000 → 0 
	0.8 * 2 = 1.60000 → 1 
	0.6 * 2 = 1.20000 → 1 
	0.2 * 2 = 0.40000 → 0 
	0.4 * 2 = 0.80000 → 0 
	0.8 * 2 = 1.60000 → 1 
	0.6 * 2 = 1.20000 → 1 
	0.2 * 2 = 0.40000 → 0 
	0.4 * 2 = 0.80000 → 0 
	0.8 * 2 = 1.60000 → 1 
	0.6 * 2 = 1.20000 → 1 
	0.2 * 2 = 0.40000 → 0 
	0.4 * 2 = 0.80000 → 0 
	\end{Verbatim}
	\hspace{1.7cm}{\texttt{Parte decimal: 0,00011001100110011001100}}			
\end{minipage}
\begin{center}
	\hspace{2cm}\boxed{\texttt{Número aproximado: 10111.00011001100110011001100}}
\end{center}

	6.Convertir los siguientes números enteros hexadecimales en sus equivalentes
	decimales:
	\begin{enumerate}
		\item C 
		\item 9F 
		\item D52 
		\item 67E 
		\item ABCD
	\end{enumerate}
	
	%66666666666666666666aaaaaaaaaaaaaaaaaaaaaaaa		
	\[
	\mathtt{
		\colorbox{yellow}{\textbf{a.} C} \qquad
		\underset{\text{F}}{\boxed{\mathtt{1111}}}
	}	
	\]
	\[
	1\times2^{3} + 1\times2^{2} + 1\times2^{1} + 1\times2^{0}
	\]
	\begin{center}
	\boxed{15_{10}}
	\end{center}
	%66666666666666666bbbbbbbbbbbbbbbbbbbbbbbbbbbb
	\[
	\mathtt{
		\colorbox{yellow}{\textbf{b.} 9F} \qquad
		\underset{\text{9}}{\boxed{\mathtt{1001}}}
		\underset{\text{F}}{\boxed{\mathtt{1111}}}
	}	
	\]
	\[
	1\times2^{7} + 0\times2^{6} + 0\times2^{5} + 1\times2^{4} + 1\times2^{3} + 1\times2^{2} + 1\times2^{1} + 1\times2^{0}
	\]
	\begin{center}
		\boxed{159_{10}}
	\end{center}
	%666666666666666666666ccccccccccccccccccccccccc
	\[
	\mathtt{
		\colorbox{yellow}{\textbf{c.} D52} \qquad
		\underset{\text{D}}{\boxed{\mathtt{1101}}}
		\underset{\text{5}}{\boxed{\mathtt{0101}}}
		\underset{\text{2}}{\boxed{\mathtt{0010}}}
	}	
	\]
	\[
	1\times2^{11} + 1\times2^{10} + 0\times2^{9} + 1\times2^{8} + 0\times2^{7} + 1\times2^{6} + 0\times2^{5} + 1\times2^{4} + 0\times2^{3} + 0\times2^{2} + 1\times2^{1} + 0\times2^{0}
	\]
	\begin{center}
		\boxed{3410_{10}}
	\end{center}
	%6666666666666666666666666ddddddddddddddddddddddddd
	\[
	\mathtt{
		\colorbox{yellow}{\textbf{d.} 67E} \qquad
		\underset{\text{6}}{\boxed{\mathtt{0110}}}
		\underset{\text{7}}{\boxed{\mathtt{0111}}}
		\underset{\text{E}}{\boxed{\mathtt{1110}}}
	}	
	\]
	\[
	0\times2^{11} + 1\times2^{10} + 1\times2^{9} + 0\times2^{8} + 0\times2^{7} + 1\times2^{6} + 1\times2^{5} + 1\times2^{4} + 1\times2^{3} + 1\times2^{2} + 1\times2^{1} + 0\times2^{0}
	\]
	\begin{center}
		\boxed{1662_{10}}
	\end{center}
	%66666666666666666666666666eeeeeeeeeeeeeeeeeeeeeeeeeeeeee
	\[
	\mathtt{
		\colorbox{yellow}{\textbf{e.} ABCD} \qquad
		\underset{\text{A}}{\boxed{\mathtt{1010}}}
		\underset{\text{B}}{\boxed{\mathtt{1011}}}
		\underset{\text{C}}{\boxed{\mathtt{1100}}}
		\underset{\text{D}}{\boxed{\mathtt{1101}}}
	}	
	\]
	\[
	1\times2^{15} + 0\times2^{14} + 1\times2^{13} + 0\times2^{12} + 1\times2^{11} + 0\times2^{10} + 1\times2^{9} + 1\times2^{8} + 1\times2^{7} + 1\times2^{6} + 0\times2^{5} + 0\times2^{4} + 1\times2^{3} + 1\times2^{2} + 0\times2^{1} + 1\times2^{0}
	\]
	\begin{center}
		\boxed{43981_{10}}
	\end{center}
	
	7. Convertir los siguientes números hexadecimales a sus equivalentes
	decimales
	\begin{enumerate} 
		\item F,4 
		\item D3,E 
		\item 111,1 
		\item 888,8 
		\item EBA,C
	\end{enumerate}	
	%77777777777777777777777777777777777aaaaaaaaaaaaaaaaaaaaaaaaaaaaaaaaaaaaaaaaa
	\[
	\mathtt{
		\colorbox{yellow}{\textbf{a.} F,4} \qquad
		\underset{\text{F}}{\boxed{\mathtt{1111}}}
		\underset{\text{,}}{\boxed{\mathtt{}}}
		\underset{\text{4}}{\boxed{\mathtt{0100}}}
	}	
	\]
	\[
	 1\times2^{3} + 1\times2^{2} + 1\times2^{1} + 1\times2^{0} + 0\times2^{-1} + 1\times2^{-2} + 0\times2^{-3} + 0\times2^{-4}
	\]
	\begin{center}
		\boxed{15,25_{10}}
	\end{center}
	%77777777777777777777777777777777777bbbbbbbbbbbbbbbbbbbbbbbbbbbbbbbbbbbbbbbbbbbbbbbbbbbbbbbbbbbbbbbbb
	\[
	\mathtt{
		\colorbox{yellow}{\textbf{b.} D3,E} \qquad
		\underset{\text{D}}{\boxed{\mathtt{1101}}}
		\underset{\text{3}}{\boxed{\mathtt{0011}}}
		\underset{\text{,}}{\boxed{\mathtt{}}}
		\underset{\text{E}}{\boxed{\mathtt{1110}}}
	}	
	\]
	\[
	1\times2^{7} + 1\times2^{6} + 0\times2^{5} + 1\times2^{4} + 0\times2^{3} + 0\times2^{2} + 1\times2^{1} + 1\times2^{0} + 1\times2^{-1} + 1\times2^{-2} + 1\times2^{-3} + 0\times2^{-4}
	\]
	\begin{center}
		\boxed{211,875_{10}}
	\end{center}
	%777777777777777777777777777777777777ccccccccccccccccccccccccccccccccccccccccccccccccccccccccccccc
	\[
	\mathtt{
		\colorbox{yellow}{\textbf{c.} 111,1} \qquad
		\underset{\text{1}}{\boxed{\mathtt{0001}}}
		\underset{\text{1}}{\boxed{\mathtt{0001}}}
		\underset{\text{1}}{\boxed{\mathtt{0001}}}
		\underset{\text{,}}{\boxed{\mathtt{}}}
		\underset{\text{1}}{\boxed{\mathtt{0001}}}
	}	
	\]
	\[
    0\times2^{11} + 0\times2^{10} + 0\times2^{9} + 1\times2^{8} + 0\times2^{7} + 0\times2^{6} + 0\times2^{5} + 1\times2^{4} + 0\times2^{3} + 0\times2^{2} + 0\times2^{1} + 1\times2^{0} + 0\times2^{-1} + 0\times2^{-2} + 0\times2^{-3} + 1\times2^{-4}
	\]
	\begin{center}
		\boxed{273,0625_{10}}
	\end{center}
	%777777777777777777777777777777777777ddddddddddddddddddddddddddddddddddddddddddddddddddddddddddd
	\[
	\mathtt{
		\colorbox{yellow}{\textbf{d.} 888,8} \qquad
		\underset{\text{8}}{\boxed{\mathtt{1000}}}
		\underset{\text{8}}{\boxed{\mathtt{1000}}}
		\underset{\text{8}}{\boxed{\mathtt{1000}}}
		\underset{\text{,}}{\boxed{\mathtt{}}}
		\underset{\text{8}}{\boxed{\mathtt{1000}}}
	}	
	\]
	\[
	1\times2^{11} + 0\times2^{10} + 0\times2^{9} + 0\times2^{8} + 1\times2^{7} + 0\times2^{6} + 0\times2^{5} + 0\times2^{4} + 1\times2^{3} + 0\times2^{2} + 0\times2^{1} + 0\times2^{0} + 1\times2^{-1} + 0\times2^{-2} + 0\times2^{-3} + 0\times2^{-4}
	\]
	\begin{center}
		\boxed{2184,5_{10}}
	\end{center}	
	%77777777777777777777777777777777777777777eeeeeeeeeeeeeeeeeeeeeeeeeeeeeeeeeeeeeeeeeeeeeeee
	\[
	\mathtt{
		\colorbox{yellow}{\textbf{e.} EBA,C} \qquad
		\underset{\text{E}}{\boxed{\mathtt{1110}}}
		\underset{\text{B}}{\boxed{\mathtt{1011}}}
		\underset{\text{A}}{\boxed{\mathtt{1010}}}
		\underset{\text{,}}{\boxed{\mathtt{}}}
		\underset{\text{C}}{\boxed{\mathtt{1100}}}
	}	
	\]
	\[
	1\times2^{11} + 1\times2^{10} + 1\times2^{9} + 0\times2^{8} + 1\times2^{7} + 0\times2^{6} + 1\times2^{5} + 1\times2^{4} + 1\times2^{3} + 0\times2^{2} + 1\times2^{1} + 0\times2^{0} + 1\times2^{-1} + 1\times2^{-2} + 0\times2^{-3} + 0\times2^{-4}
	\]
	\begin{center}
		\boxed{3770,75_{10}}
	\end{center}	

	8. Convertir los números (AF315)16 y (7326)8 a base 10 y base 2	
	
	\[
	\mathtt{
		\colorbox{yellow}{ $AF315_{16}$} \qquad
		\underset{\text{A}}{\boxed{\mathtt{1010}}}
		\underset{\text{F}}{\boxed{\mathtt{1111}}}
		\underset{\text{3}}{\boxed{\mathtt{0011}}}
		\underset{\text{1}}{\boxed{\mathtt{0001}}}
		\underset{\text{5}}{\boxed{\mathtt{0101}}}
	}	
	\]
	\[
	1\times2^{19} + 0\times2^{18} + 1\times2^{17} + 0\times2^{16} + 1\times2^{15} + 1\times2^{14} + 1\times2^{13} + 1\times2^{12} + 0\times2^{11} + 0\times2^{10} + 1\times2^{9} + 1\times2^{8} + 0\times2^{7} + 0\times2^{6} + 0\times2^{5} + 1\times2^{4} + 0\times2^{3} + 1\times2^{2} + 0\times2^{1} + 1\times2^{0}
	\]
	\begin{center}
		\boxed{717589_{10}}
	\end{center}
	
	
	\[
	\mathtt{
		\colorbox{yellow}{ $7326_{8}$} \qquad
		\underset{\text{7}}{\boxed{\mathtt{111}}}
		\underset{\text{3}}{\boxed{\mathtt{011}}}
		\underset{\text{2}}{\boxed{\mathtt{010}}}
		\underset{\text{6}}{\boxed{\mathtt{110}}}
	}	
	\]
	\[
  	 1\times2^{11} + 1\times2^{10} + 1\times2^{9} + 0\times2^{8} + 1\times2^{7} + 1\times2^{6} + 0\times2^{5} + 1\times2^{4} + 0\times2^{3} + 1\times2^{2} + 1\times2^{1} + 0\times2^{0}
	\]
	\begin{center}
		\boxed{3798_{10}}
	\end{center}

	9. Convertir los números (245,625)10 y (1797,223)10 a binario, octal y hexadecimal.
	
	\begin{center}
		\colorbox{yellow}{{$245,625_{10}$}} \\ \vspace{0.3cm}
	\end{center}	
	\vspace{-1em}
	%ACA EMPIEZA EL MINIPAGE
	\begin{minipage}[t]{0.40\textwidth}
		\hspace{2,5cm}\text{Parte entera: 245}
		
\begin{Verbatim}
           245 / 2 = 122, residuo 1
  	 122 / 2 =  61, residuo 0
 	   61 / 2 =  30, residuo 1
 	   30 / 2 =  15, residuo 0
 	   15 / 2 =   7, residuo 1
             7 / 2 =   3, residuo 1
             3 / 2 =   1, residuo 1
	     1 -----------------> 1
	     
	    Lectura inversa: 11110101
		\end{Verbatim}
	
	\end{minipage}
	\hfill
	\begin{minipage}[t]{0.625\textwidth}
		\hspace{2cm}\text{Parte decimal: 0,1}
		
		\begin{Verbatim}
			0.625 * 2 = 1.25 → 1 
			0.250 * 2 = 0.50 → 0 
			0.500 * 2 = 1.00 → 1 
			0.000 * 2 = 0.00 → 0 
		\end{Verbatim}
		\hspace{5cm}{\texttt{Parte decimal: 0,101}}			
	\end{minipage}
	\begin{center}
		\hspace{2cm}\boxed{\texttt{Número completo binario: $11110101.101_2$}}
	\end{center}
	
	\[
	\mathtt{
		\colorbox{yellow}{Número en octal:} \qquad
		\underset{3}{\boxed{\mathtt{011}}}
		\underset{6}{\boxed{\mathtt{110}}}
		\underset{5}{\boxed{\mathtt{101}}}
		\underset{}{\boxed{\mathtt{,}}}
		\underset{5}{\boxed{\mathtt{101}}}
		\text{ → $365,5_8$}
	}	
	\]
	
	\[
	\mathtt{
		\colorbox{yellow}{Número en hexadecimal:} \qquad
		\underset{F}{\boxed{\mathtt{1111}}}
		\underset{5}{\boxed{\mathtt{0101}}}
		\underset{}{\boxed{\mathtt{,}}}
		\underset{A}{\boxed{\mathtt{1010}}}
		\text{ → $F5,A_{16}$}
	}	
	\]
	
%el segundo el segundo el segundo el segundo el segundo el segundo

\begin{center}
	\colorbox{yellow}{{$1797,223_{10}$}} \\ \vspace{0.3cm}
\end{center}	
\vspace{-1em}
%ACA EMPIEZA EL MINIPAGE
\begin{minipage}[t]{0.40\textwidth}
	\hspace{2,5cm}\text{Parte entera: 245}
	
	\begin{Verbatim}
	1797 / 2 = 898, residuo 1
         898 / 2 = 449, residuo 0
	 249 / 2 = 224, residuo 1
	 224 / 2 = 112, residuo 0
	 112 / 2 =  56, residuo 0
	  56 / 2 =  28, residuo 0
	  28 / 2 =  14, residuo 0
	  14 / 2 =   7, residuo 0
	   7 / 2 =   3, residuo 1
	   3 / 2 =   1, residuo 1
	   1 -----------------> 1
		
	Lectura inversa: 11100000101
	\end{Verbatim}
	
\end{minipage}
\hfill
\begin{minipage}[t]{0.625\textwidth}
	\hspace{2cm}\text{Parte decimal: 0,1}
	
	\begin{Verbatim}
		0.22300 * 2 = 0.44600 → 0 
		0.44600 * 2 = 0.89200 → 0 
		0.89200 * 2 = 1.78400 → 1  
		0.78400 * 2 = 1.56800 → 1 
		0.56800 * 2 = 1.13600 → 1 
		0.13600 * 2 = 0.27200 → 0 
		0.27200 * 2 = 0.54400 → 0 
		0.54400 * 2 = 1.08800 → 1 
		0.08800 * 2 = 0.17600 → 0 
		0.17600 * 2 = 0.35200 → 0 
		0.35200 * 2 = 0.70400 → 0 
		0.70400 * 2 = 1.40800 → 1 
		0.40800 * 2 = 0.81600 → 0 
		0.81600 * 2 = 1.63200 → 1 
		0.63200 * 2 = 1.26400 → 1 
		0.26400 * 2 = 0.52800 → 0 
		0.52800 * 2 = 1.05600 → 1 
		0.05600 * 2 = 0.11200 → 0 
		0.11200 * 2 = 0.22400 → 0 
		0.22400 * 2 = 0.44800 → 0 
		0.44800 * 2 = 0.89600 → 0 
		0.89600 * 2 = 1.79200 → 1 
		0.79200 * 2 = 1.58400 → 1 
		
	Parte decimal: 0,00111001000101101000011
	\end{Verbatim} 
	
\end{minipage}
\begin{center}
	\hspace{2cm}\boxed{\texttt{Número aproximado: 11100000101.00111001000101101000011}}
\end{center}

\[
\mathtt{
	\colorbox{yellow}{Número en octal:} \qquad
	\underset{3}{\boxed{\mathtt{011}}}
	\underset{4}{\boxed{\mathtt{100}}}
	\underset{0}{\boxed{\mathtt{000}}}
	\underset{5}{\boxed{\mathtt{101}}}
	\underset{}{\boxed{\mathtt{,}}}
	\underset{1}{\boxed{\mathtt{001}}}
	\underset{6}{\boxed{\mathtt{110}}}
	\underset{2}{\boxed{\mathtt{010}}}
	\underset{1}{\boxed{\mathtt{001}}}
	\underset{3}{\boxed{\mathtt{011}}}
	\underset{2}{\boxed{\mathtt{010}}}
	\text{ → $3405.162132_8$}
}	
\] 

\[
\mathtt{
	\colorbox{yellow}{Número en hexadecimal:} \qquad
	\underset{7}{\boxed{\mathtt{0111}}}
	\underset{0}{\boxed{\mathtt{0000}}}
	\underset{5}{\boxed{\mathtt{0101}}}
	\underset{}{\boxed{\mathtt{,}}}
	\underset{3}{\boxed{\mathtt{0011}}}
	\underset{9}{\boxed{\mathtt{1001}}}
	\underset{1}{\boxed{\mathtt{0001}}}
	\underset{6}{\boxed{\mathtt{0110}}}
	\underset{8}{\boxed{\mathtt{1000}}}
	\text{ → $705.39168_{16}$}
}	
\]	
	
	10.Convertir el número (49403180,AF7)16 a binario, octal y decimal. 
	\begin{center}
	\colorbox{yellow}{{$49403180,AF7_{16}$}} \\ \vspace{0.3cm}

	\text{Forma binaria: Cada numero hexadecimal formado por binarios de 4 bits}
	\[
	\mathtt{
		\underset{4}{\boxed{\mathtt{0100}}}
		\underset{9}{\boxed{\mathtt{1001}}}
		\underset{4}{\boxed{\mathtt{0100}}}
		\underset{0}{\boxed{\mathtt{0000}}}
		\underset{3}{\boxed{\mathtt{0011}}}
		\underset{1}{\boxed{\mathtt{0001}}}
		\underset{8}{\boxed{\mathtt{1000}}}
		\underset{0}{\boxed{\mathtt{0000}}}
		\underset{}{\boxed{\mathtt{,}}}
		\underset{A}{\boxed{\mathtt{1010}}}
		\underset{F}{\boxed{\mathtt{1111}}}
		\underset{7}{\boxed{\mathtt{0111}}}
	}	
	\]
	\text{Forma Octal: Separando el binario cada 3 bits para cada cifra}
	\[
	\mathtt{
		\underset{4}{\boxed{\mathtt{0100}}}
		\underset{9}{\boxed{\mathtt{1001}}}
		\underset{4}{\boxed{\mathtt{0100}}}
		\underset{0}{\boxed{\mathtt{0000}}}
		\underset{3}{\boxed{\mathtt{0011}}}
		\underset{1}{\boxed{\mathtt{0001}}}
		\underset{8}{\boxed{\mathtt{1000}}}
		\underset{0}{\boxed{\mathtt{0000}}}
		\underset{}{\boxed{\mathtt{,}}}
		\underset{A}{\boxed{\mathtt{1010}}}
		\underset{F}{\boxed{\mathtt{1111}}}
		\underset{7}{\boxed{\mathtt{0111}}}
	}	
	\]
	\text{Forma decimal: Usando el binario mediante el método fundamental de la numeración}
	\begin{multline*}
		0\times2^{31} + 1\times2^{30} + 0\times2^{29} + 0\times2^{28} + 1\times2^{27} + 0\times2^{26} + 0\times2^{25} + 1\times2^{24} +	\\0\times2^{23} + 1\times2^{22} + 0\times2^{21} + 0\times2^{20} + 0\times2^{19} + 0\times2^{18} + 0\times2^{17} + 0\times2^{16} +	\\0\times2^{15} + 0\times2^{14} + 1\times2^{13} + 1\times2^{12} + 0\times2^{11} + 0\times2^{10} + 0\times2^{9} + 1\times2^{8} +\\1\times2^{7} + 0\times2^{6} + 0\times2^{5} + 0\times2^{4} + 0\times2^{3} + 0\times2^{2} + 0\times2^{1} + 0\times2^{0} + \\1\times2^{-1} + 0\times2^{-2} + 1\times2^{-3} + 0\times2^{-4} + 1\times2^{-5} + 1\times2^{-6} + 1\times2^{-7} + 1\times2^{-8} +\\0\times2^{-9} + 1\times2^{-10} + 1\times2^{-11} + 1\times2^{-12}
	\end{multline*}
\end{center}	

	\hspace{5cm}\boxed{\text{Resultado: $1228943744.6853027_{10}$}} \\


	11. Convertir los siguientes números de base 10 a base 2,base 8 y base 16
	\begin{enumerate}
		\item  13 
		\item  94 
		\item 356
	\end{enumerate}
\begin{center}	
	\colorbox{yellow}{{\textbf{a.} $13$}} \\ \vspace{0.3cm}
		     \begin{Verbatim}
		     	13 / 2 =  6, residuo 1
		     	 6 / 2 =  3, residuo 0
		     	 3 / 2 =  1, residuo 1
		     	 1 ----------------> 1
		     	
		     	 Lectura inversa: 1101
		     \end{Verbatim}
	     
	     \[
	     \mathtt{
	     	\colorbox{yellow}{Número en octal:} \qquad
	     	\underset{1}{\boxed{\mathtt{001}}}
	     	\underset{5}{\boxed{\mathtt{101}}}
	     	\text{ → $15_8$}
	     }	
	     \] 
	     
	     \[
	     \mathtt{
	     	\colorbox{yellow}{Número en hexadecimal:} \qquad
	     	\underset{D}{\boxed{\mathtt{1101}}}
	      	\text{ → $D_{16}$}
	     }	
	     \]	
\end{center}
\begin{center}	
	\colorbox{yellow}{{\textbf{b.} $94$}} \\ \vspace{0.3cm}
		    \begin{Verbatim}
		    	94 / 2 =  47, residuo 0
		    	47 / 2 =  23, residuo 1
		    	23 / 2 =  11, residuo 1
		    	11 / 2 =   5, residuo 1
		             5 / 2 =   2, residuo 1
		    	 2 / 2 =   1, residuo 0
		    	 1 -----------------> 1
		    	
		    	Lectura inversa: 1011110
		    \end{Verbatim}
	
	\[
	\mathtt{
		\colorbox{yellow}{Número en octal:} \qquad
		\underset{1}{\boxed{\mathtt{001}}}
		\underset{3}{\boxed{\mathtt{011}}}
		\underset{6}{\boxed{\mathtt{110}}}
		\text{ → $136_8$}
	}	
	\] 
	
	\[
	\mathtt{
		\colorbox{yellow}{Número en hexadecimal:} \qquad
		\underset{5}{\boxed{\mathtt{0101}}}
		\underset{E}{\boxed{\mathtt{1110}}}
		\text{ → $5E_{16}$}
	}	
	\]		
\end{center}	
\begin{center}	
	\colorbox{yellow}{{\textbf{c.} $356$}} \\ \vspace{0.3cm}
	    	\begin{Verbatim}
	    		356 / 2 = 178, residuo 0
	    		178 / 2 =  89, residuo 0
	    		 89 / 2 =  44, residuo 1
	    	         44 / 2 =  22, residuo 0
	    		 22 / 2 =  11, residuo 0
	    		 11 / 2 =   5, residuo 1
	    	 	 5 / 2 =   2, residuo 1
	    		  2 / 2 =   1, residuo 0
	    		  1 -----------------> 1
		
	        	    Lectura inversa: 101100100
	\end{Verbatim}
	
	\[
	\mathtt{
		\colorbox{yellow}{Número en octal:} \qquad
		\underset{5}{\boxed{\mathtt{101}}}
		\underset{4}{\boxed{\mathtt{100}}}
		\underset{4}{\boxed{\mathtt{100}}}
		\text{ → $544_8$}
	}	
	\] 
	
	\[
	\mathtt{
		\colorbox{yellow}{Número en hexadecimal:} \qquad
		\underset{1}{\boxed{\mathtt{0001}}}
		\underset{6}{\boxed{\mathtt{0110}}}
		\underset{4}{\boxed{\mathtt{0100}}}
		\text{ → $164_{16}$}
	}	
	\]		
\end{center}	
	
	12.Convertir los siguientes números de base 10 a base 2.
	\begin{enumerate}
		\item 0,00625 
		\item 43,32 
		\item 0,51	
	\end{enumerate}
	
	\begin{center}	
		\colorbox{yellow}{{\textbf{a.} $0,00625$}} \\ \vspace{0.3cm}
		
		Parte entera = 0
		
        	\begin{Verbatim}
        		0.00625 * 2 = 0.01250 → 0 
        		0.01250 * 2 = 0.02500 → 0 
        		0.02500 * 2 = 0.05000 → 0 
        		0.05000 * 2 = 0.10000 → 0 
        		0.10000 * 2 = 0.20000 → 0 
        		0.20000 * 2 = 0.40000 → 0 
        		0.40000 * 2 = 0.80000 → 0 
        		0.80000 * 2 = 1.60000 → 1 
        		0.60000 * 2 = 1.20000 → 1 
        		0.20000 * 2 = 0.40000 → 0 
        		0.40000 * 2 = 0.80000 → 0 
        		0.80000 * 2 = 1.60000 → 1 
        		0.60000 * 2 = 1.20000 → 1 
        		0.20000 * 2 = 0.40000 → 0 
        		0.40000 * 2 = 0.80000 → 0 
        		0.80000 * 2 = 1.60000 → 1 
        		0.60000 * 2 = 1.20000 → 1 
        		0.20000 * 2 = 0.40000 → 0 
        		0.40000 * 2 = 0.80000 → 0 
        		0.80000 * 2 = 1.60000 → 1 
        		0.60000 * 2 = 1.20000 → 1 
        		0.20000 * 2 = 0.40000 → 0 
        		0.40000 * 2 = 0.80000 → 0
        		
        		Parte decimal: 00000001100110011001100
        	\end{Verbatim}
		\boxed{\text{Numero aproximado: $0,00000001100110011001100_2$}}
	\end{center}
	\begin{center}
	\colorbox{yellow}{{\textbf{b.} $43,32$}} \\ \vspace{0.3cm}
	             \begin{Verbatim}
	             	43 / 2 =  21, residuo 1
	             	21 / 2 =  10, residuo 1
	             	10 / 2 =   5, residuo 0
	             	 5 / 2 =   2, residuo 1
	             	 2 / 2 =   1, residuo 0
	             	 1 -----------------> 1
	             	
	             	Lectura inversa: 101011
	             \end{Verbatim}
	
	    		\begin{Verbatim}
	    		0.32000 * 2 = 0.64000 → 0 
	    		0.64000 * 2 = 1.28000 → 1 
	    		0.28000 * 2 = 0.56000 → 0 
	    		0.56000 * 2 = 1.12000 → 1 
	    		0.12000 * 2 = 0.24000 → 0 
	    		0.24000 * 2 = 0.48000 → 0 
	    		0.48000 * 2 = 0.96000 → 0 
	    		0.96000 * 2 = 1.92000 → 1 
	    		0.92000 * 2 = 1.84000 → 1 
	    		0.84000 * 2 = 1.68000 → 1 
	    		0.68000 * 2 = 1.36000 → 1 
	    		0.36000 * 2 = 0.72000 → 0 
	    		0.72000 * 2 = 1.44000 → 1 
	    		0.44000 * 2 = 0.88000 → 0 
	    		0.88000 * 2 = 1.76000 → 1 
	    		0.76000 * 2 = 1.52000 → 1 
	    		0.52000 * 2 = 1.04000 → 1 
	    		0.04000 * 2 = 0.08000 → 0 
	    		0.08000 * 2 = 0.16000 → 0 
	    		0.16000 * 2 = 0.32000 → 0 
	    		0.32000 * 2 = 0.64000 → 0 
	    		0.64000 * 2 = 1.28000 → 1 
	    		0.28000 * 2 = 0.56000 → 0
	    
	    	Parte decimal: 01010001111010111000010
	    \end{Verbatim}
	
		\boxed{\text{Numero aproximado: $101011,01010001111010111000010_2$}}
	\end{center}
	\begin{center}	
		\colorbox{yellow}{{\textbf{c.} $0,51$}} \\ \vspace{0.3cm}
		
		Parte entera = 0
		
		\begin{Verbatim}
			0.51000 * 2 = 1.02000 → 1 
			0.02000 * 2 = 0.04000 → 0 
			0.04000 * 2 = 0.08000 → 0 
			0.08000 * 2 = 0.16000 → 0 
			0.16000 * 2 = 0.32000 → 0 
			0.32000 * 2 = 0.64000 → 0 
			0.64000 * 2 = 1.28000 → 1 
			0.28000 * 2 = 0.56000 → 0 
			0.56000 * 2 = 1.12000 → 1 
			0.12000 * 2 = 0.24000 → 0 
			0.24000 * 2 = 0.48000 → 0 
			0.48000 * 2 = 0.96000 → 0 
			0.96000 * 2 = 1.92000 → 1 
			0.92000 * 2 = 1.84000 → 1 
			0.84000 * 2 = 1.68000 → 1 
			0.68000 * 2 = 1.36000 → 1 
			0.36000 * 2 = 0.72000 → 0 
			0.72000 * 2 = 1.44000 → 1 
			0.44000 * 2 = 0.88000 → 0 
			0.88000 * 2 = 1.76000 → 1 
			0.76000 * 2 = 1.52000 → 1 
			0.52000 * 2 = 1.04000 → 1 
			0.04000 * 2 = 0.08000 → 0
			
		Parte decimal: 10000010100011110101110
		\end{Verbatim}
		\boxed{\text{Numero aproximado: $0,10000010100011110101110_2$}}
	\end{center}
	
	13.Escribir el equivalente de base 8 de los siguientes números en base 2.
	\begin{enumerate}
		\item 10111100101 
		\item 1101,101
		\item 1,0111
	\end{enumerate}
	
	\begin{center}
	\colorbox{yellow}{{\textbf{a.} $10111100101_2$}} \\ 
	\[
	1\times2^{10} + 0\times2^{9} + 1\times2^{8} + 1\times2^{7} + 1\times2^{6} + 1\times2^{5} + 0\times2^{4} + 0\times2^{3} + 1\times2^{2} + 0\times2^{1} + 1\times2^{0}
	\]
	\boxed{1509_{10}}
	\end{center}
	
		\begin{center}
		\colorbox{yellow}{{\textbf{b.} $1101,101_2$}} \\ 
		\[
		1\times2^{3} + 1\times2^{2} + 0\times2^{1} + 1\times2^{0} + 1\times2^{-1} + 0\times2^{-2} + 1\times2^{-3}
		\]
		\boxed{13,625_{10}}
	\end{center}
	
	\begin{center}
		\colorbox{yellow}{{\textbf{c.} $1,0111_2$}} \\ 
		\[
		1\times2^{0} + 0\times2^{-1} + 1\times2^{-2} + 1\times2^{-3} + 1\times2^{-4} 
		\]
		\boxed{1,4375_{10}}
	\end{center}
	
	14. Calcular el valor decimal de los números binarios (11100111) y (10111111)
	suponiendo que están representados en complemento a 2. Repetir el ejercicio
	suponiendo que están representados en complemento a 1.
	
	\begin{center}
	\subsection*{Complemento a2: 11100111}
	\[
	\text{Invertir bits: } 00011000
	\]
	\[
	\text{Sumar 1: } 00011000 + 1 = 00011001 = 25
	\]
	\[
	\boxed{\text{Resultado final: } -25}
	\]
	
	\subsection*{Complemento a2: 10111111}
	\[
	\text{Invertir bits: } 01000000
	\]
	\[
	\text{Sumar 1: } 01000000 + 1 = 01000001 = 65
	\]
	\[
	\boxed{\text{Resultado final: } -65}
	\]	
	
	\subsection*{Complemento a1: 11100111}
	\[
	\text{Invertir bits: } 00011000 = 24
	\]
	\[
	\boxed{\text{Resultado final: } -24}
	\]
	
	\subsection*{Complemento a1: 10111111}
	\[
	\text{Invertir bits: } 01000000 = 64
	\]
	\[
	\boxed{\text{Resultado final: } -64}
	\]
	\end{center}
	
	15.Resolver los ejercicios siguientes:
	\begin{enumerate}
		\item Representar (-499)10 en magnitud y signo.
		\item Representar (-628)10 en complemento a 2.
		\item Convertir a base 10 el número binario 1001000110, dado en magnitud y signo.
		\item Convertir a base 10 el número binario 1110011101, dado en complemento a2
	\end{enumerate}
	
	\begin{center}
	\colorbox{yellow}{{\textbf{a.} $-499$}} \\ \vspace{0.3cm}
		\begin{Verbatim}
			499 / 2 =  249, residuo 1
			249 / 2 =  124, residuo 1
			124 / 2 =   62, residuo 0
			 62 / 2 =   31, residuo 0
			 31 / 2 =   15, residuo 1
			 15 / 2 =    7, residuo 1
			  7 / 2 =    3, residuo 1
			  3 / 2 =    1, residuo 1
			  1 ------------------> 1
		
	      	Lectura inversa: 111110011
	\end{Verbatim}
	\[
	\text{Bit de signo: } 1
	\]
	\[
	\text{Magnitud binario: }  111110011
	\]
	\[
	\boxed{\text{Resultado final: } 1111110011}
	\]
	\end{center}
	
		\begin{center}
		\colorbox{yellow}{{\textbf{b.} $-628$}} \\ \vspace{0.3cm}
		\begin{Verbatim}
			628 / 2 =  314, residuo 0
			314 / 2 =  157, residuo 0
			157 / 2 =   78, residuo 1
			 78 / 2 =   39, residuo 0
			39 / 2 =   19, residuo 1
			19 / 2 =    9, residuo 1
			9 / 2 =    4, residuo 1
			4 / 2 =    2, residuo 0
			2 / 2 =    1, residuo 0
			1 ------------------> 1
			
			Lectura inversa: 1001110100
		\end{Verbatim}
		\[
		\text{Se le agrega un bit mas para alcanzar el rango: } 01001110100
		\]
		\[
		\text{Invertir bits: } 10110001011
		\]
		\[
		\text{Sumar 1: } 10110001011 + 1 = 10110001100
		\]
		\[
		\boxed{\text{Resultado final: } 10110001100}
		\]
	\end{center}
	
	\begin{center}
		\colorbox{yellow}{{\textbf{c.} $1001000110_2$}} \\
		\[
		\text{Primer bit de signo es negativo: } 1
		\] 
		\[
		0\times2^{8} + 0\times2^{7} + 1\times2^{6} + 0\times2^{5} + 0\times2^{4} + 0\times2^{3} + 1\times2^{2} + 1\times2^{1} + 0\times2^{0}
		\]
		\boxed{-70_{10}}
	\end{center}
	
	\begin{center}
	\colorbox{yellow}{{\textbf{d.} $1110011101_2$}} \\
	\[
	\text{Invertir bits: } 0001100010
	\]
	\[
	\text{Sumar 1: } 0001100010 + 1 = 0001100011 = 99
	\]
	\[
	\boxed{\text{Resultado final: } -99}
	\]
	\end{center}
	
	
\end{document}