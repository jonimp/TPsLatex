\documentclass{article}
\usepackage[edges]{forest}
\usepackage{amsmath}
\usepackage{listings}
\usepackage{fancyhdr}
\usepackage{fancyvrb}
\setlength{\headheight}{15pt}
\usepackage{caption}
\usepackage{graphicx}
\usepackage{xcolor}
\usepackage{xparse}
\usepackage{enumitem}
\usepackage{float}
\usepackage{textcomp}
\usepackage{tikz}
\usepackage[a4paper, top=2.5cm, bottom=2cm, left=2cm, right=2cm]{geometry}
\usepackage{emoji}
\usepackage{booktabs}
\usepackage{array}
\usepackage{tabularx}

\pagestyle{fancy}
\fancyhf{}
\rhead{INSPT - UTN}
\lhead{Jonatan Imperi}
\cfoot{\thepage}
\renewcommand{\theenumi}{\alph{enumi}}

\NewDocumentCommand{\Xor}{m m m o}{
	\begin{minipage}{0.35\linewidth}
		\centering
		\[
		\renewcommand{\arraystretch}{1.2}
		\begin{array}{l@{\quad}l}
			& \texttt{#1} \\
			& \texttt{#2} \\
			\text{Xor} & \texttt{#3} \\
			\hline
			& \IfValueT{#4}{\texttt{#4}}
		\end{array}
		\]
	\end{minipage}
}



\begin{document}
	\begin{center}
		
		\LARGE \textbf{Sistemas de computación 1} \\[0.5cm]

		\LARGE Trabajo práctico n° 4 \\
	\end{center}	
	
	\section*{Código de Hamming y código de Huffmann}
	
	1. Obtener la codificación de Hauffman, el árbol binario y el porcentaje de compresión
	de los siguientes strings:
	\begin{enumerate}
	\item hauffmanyhamming
	\item hola mundo
	\item dddrdddrdrr
	\item anita lava la tina
	\item ccbcbbbcbbb
	\item FFCE FEEF EFED EFFF EDEF EEFE DEFE FFFF FDFF BFFC FDFF FEAE DCDE
	\item (tu nombre y apellido) Ej: maria laura frette
	\item CATCATCATCAT
	\item \emoji{red-heart}\emoji{red-heart}\emoji{red-heart}\emoji{red-heart}\emoji{red-heart}\emoji{red-heart}\emoji{soccer-ball}\emoji{soccer-ball}\emoji{soccer-ball}\emoji{soccer-ball}\emoji{red-triangle-pointed-up}\emoji{red-triangle-pointed-up}\emoji{red-triangle-pointed-up}\emoji{cross-mark}\emoji{cross-mark}\emoji{cross-mark}\emoji{cross-mark}\emoji{cross-mark}\emoji{cross-mark}
	\item \emoji{crab}\emoji{crab}\emoji{crab}\emoji{crab}\emoji{crab}\emoji{crab}\emoji{crab}\emoji{christmas-tree}\emoji{christmas-tree}\emoji{christmas-tree}\emoji{mouse-face}\emoji{mouse-face}\emoji{dog-face}\emoji{dog-face}\emoji{dog-face}\emoji{dog-face}\emoji{dog-face} 
	\end{enumerate}

	\vspace{0.5cm}
	\begin{center}
	\hspace{3cm}\colorbox{yellow}{{\textbf{a.} hauffmanyhamming}}\newline
	\subsection*{Total: 15 caracteres x 7 bits = 105 bits}
	\vspace{0.5cm}
	\subsection*{Conteo de caracteres}
\begin{tabular}{c|c}
	\textbf{Caracter} & \textbf{Frecuencia} \\
	\hline
	a & 3 \\
	m & 2 \\
	h & 2 \\
	n & 2 \\
	f & 2 \\
	u & 1 \\
	y & 1 \\
	i & 1 \\
	g & 1 \\
\end{tabular}

	\vspace{1cm}
	
	\begin{forest}
		[N8:16, for tree={parent anchor=south, child anchor=north, draw, l sep=10mm, s sep=5mm, minimum size=8mm} 
			[N6:7, edge label={node[midway, left]{0}}
				[a,tier=word, edge label={node[midway, left]{0}}]
				[N3:4, edge label={node[midway, right]{1}} [f,tier=word, edge label={node[midway, left]{0}}] [n,tier=word, edge label={node[midway, right]{1}}]]
			]
			[N7:9, edge label={node[midway, right]{1}}
				[N4:4, edge label={node[midway, left]{0}}[h,tier=word, edge label={node[midway, left]{0}}]
				[N1:2, edge label={node[midway, right]{1}}[i,tier=word, edge label={node[midway, left]{0}}][g,tier=word, edge label={node[midway, right]{1}}], edge label={node[midway, right]{1}}]
				]
				[N5:5, edge label={node[midway, right]{1}}[m,tier=word, edge label={node[midway, left]{0}}][N2:2, edge label={node[midway, right]{1}}[u,tier=word, edge label={node[midway, left]{0}}][y,tier=word, edge label={node[midway, right]{1}}]]]
				]
			]
			]
	\end{forest}
	
	\subsection*{Codificación resultante}
	\begin{tabular}{c|c|c|c}
		\textbf{Caracter} & \textbf{Frecuencia} & \textbf{Código} & \textbf{Uso de bits}\\
		\hline
		a & 3 & 00   & 6\\
		m & 3 & 110  & 9\\
		h & 2 & 100  & 6\\
		n & 2 & 011  & 6\\
		f & 2 & 010  & 6\\
		u & 1 & 1110 & 4\\
		y & 1 & 1111 & 4\\
		i & 1 & 1010 & 4\\
		g & 1 & 1011 & 4\\
	\end{tabular}
	\subsection*{Tasa de compresión: 105 bits / 49 bits = 2.14}	
	\end{center}
	
	%BBBBBBBBBBBBBBBBBBBBBBBBBBBBBBBBBBBBBBBBBBBBBBBBBBBBBBBBBBB
	\vspace{0.5cm}
	\begin{center}
		\hspace{3cm}\colorbox{yellow}{{\textbf{b.} hola mundo}}\newline
		\subsection*{Total: 10 caracteres x 7 bits = 70 bits}
		\vspace{0.5cm}
		\subsection*{Conteo de caracteres}
		\begin{tabular}{c|c}
			\textbf{Caracter} & \textbf{Frecuencia} \\
			\hline
			o & 2 \\
			h & 1 \\
			l & 1 \\
			a & 1 \\
			m & 1 \\
			u & 1 \\
			n & 1 \\
			d & 1 \\
			  & 1 \\
		\end{tabular}
		
		\vspace{1cm}
		
		\begin{forest}
			for tree={
				parent anchor=south,
				child anchor=north,
				draw,
				l sep=10mm,
				s sep=5mm,
				minimum size=8mm
			}
			[N8:10
				[N7:6, edge label={node[midway, left]{0}}
					[N1:2, edge label={node[midway, left]{0}}
						[h, tier=word, edge label={node[midway, left]{0}}]
						[l, tier=word, edge label={node[midway, right]{1}}]	
					]
					[N6:4, edge label={node[midway,right]{1}}
						[N2:2, edge label={node[midway,left]{0}}
							[a, tier=word, edge label={node[midway, left]{0}}]
							[ , tier=word, edge label={node[midway, right]{1}}]	
						]
						[N3:2, edge label={node[midway,right]{1}}
							[m, tier=word, edge label={node[midway, left]{0}}]
							[u, tier=word, edge label={node[midway, right]{1}}]	
						]
					]
				]
				[N5:4, edge label={node[midway,right]{1}}
					[N4:2,edge label={node[midway,left]{0}}
						[n, tier=word, edge label={node[midway, left]{0}}]
						[d, tier=word, edge label={node[midway, right]{1}}]
					]
					[o, tier=word, edge label={node[midway, right]{1}}]
				]
			]
		\end{forest}
		
		
		\subsection*{Codificación resultante}
		\begin{tabular}{c|c|c|c}
			\textbf{Caracter} & \textbf{Frecuencia} & \textbf{Código} & \textbf{Uso de bits}\\
			\hline
			o & 2 & 11    & 4\\
			h & 1 & 000   & 3\\
			l & 1 & 001   & 3\\
			a & 1 & 0100  & 4\\
			m & 1 & 0110  & 4\\
			u & 1 & 0111  & 4\\
			n & 1 & 100   & 4\\
			d & 1 & 101   & 4\\
			  & 1 & 0101  & 4\\
		\end{tabular}
		\subsection*{Tasa de compresión: 70 bits / 34 bits = 2.05}	
	\end{center}
	
	%CCCCCCCCCCCCCCCCCCCCCCCCCCCCCCCCCCCCCCCCCCCCCCCCCCCCCCCCCCCCCCCCCCCCC
	\vspace{0.5cm}
	\begin{center}
		\hspace{3cm}\colorbox{yellow}{{\textbf{c.} dddrdddrdrr}}\newline
		\subsection*{Total: 11 caracteres x 7 bits = 77 bits}
		\vspace{0.5cm}
		\subsection*{Conteo de caracteres}
		\begin{tabular}{c|c}
			\textbf{Carácter} & \textbf{Frecuencia} \\
			\hline
			d & 7 \\
			r & 4 \\
		\end{tabular}
		
		\vspace{1cm}
		
		\begin{forest}
			for tree={
				parent anchor=south,
				child anchor=north,
				draw,
				l sep=10mm,
				s sep=5mm,
				minimum size=8mm
			}
			[N1:11
				[d, tier=word, edge label={node[midway, left]{0}}]
				[r, tier=word, edge label={node[midway, right]{1}}]
			]
		\end{forest}
		
		
		\subsection*{Codificación resultante}
		\begin{tabular}{c|c|c|c}
			\textbf{Caracter} & \textbf{Frecuencia} & \textbf{Código} & \textbf{Uso de bits}\\
			\hline
			d & 7 & 0 & 7\\
			r & 4 & 1 & 4\\
		\end{tabular}
		\subsection*{Tasa de compresión: 77 bits / 11 bits = 7}	
	\end{center}
	
	%DDDDDDDDDDDDDDDDDDDDDDDDDDDDDDDDDDDDDDDDDDDDDDDDDDDDDDDDDDDDDDDDDD
	\vspace{0.5cm}
	\begin{center}
		\hspace{3cm}\colorbox{yellow}{{\textbf{d.} anita lava la tina}}\newline
		\subsection*{Total: 18 caracteres x 7 bits = 126 bits}
		\vspace{0.5cm}
		\subsection*{Conteo de caracteres}
		\begin{tabular}{c|c}
			\textbf{Carácter} & \textbf{Frecuencia} \\
			\hline
			a & 6 \\
			l & 2 \\
			  & 3 \\
			n & 2 \\
			i & 2 \\
			t & 2 \\
			v & 1 \\  
		\end{tabular}
		
		\vspace{1cm}
		
		\begin{forest}
			for tree={
				parent anchor=south,
				child anchor=north,
				draw,
				l sep=10mm,
				s sep=5mm,
				minimum size=8mm
			}
			[N6:18
				[N4:7, edge label={node[midway, left]{0}}
					[N1:3, edge label={node[midway,left]{0}}
						[v, tier=word, edge label={node[midway, left]{0}}]
						[t, tier=word, edge label={node[midway, right]{1}}]
					]
					[N2:4, edge label={node[midway,right]{1}}
					[i, tier=word, edge label={node[midway, left]{0}}]
					[n, tier=word, edge label={node[midway, right]{1}}]
					]
				]
				[N5:11, edge label={node[midway, right]{1}}
					[N3:5, edge label={node[midway,left]{0}}
						[l, tier=word, edge label={node[midway, left]{0}}]
						[ , tier=word, edge label={node[midway, right]{1}}]
				]
					[a, tier=word, edge label={node[midway, right]{1}}]
				]
			]
		\end{forest}
		
		
		\subsection*{Codificación resultante}
		\begin{tabular}{c|c|c|c}
			\textbf{Caracter} & \textbf{Frecuencia} & \textbf{Código} & \textbf{Uso de bits}\\
			\hline
			a & 6 & 11 & 12\\
			l & 2 & 100 & 6\\
			  & 3 & 101 & 9\\
			n & 2 & 011 & 6\\
			i & 2 & 010 & 6\\
			t & 2 & 001 & 6\\
			v & 1 & 000 & 3\\
		\end{tabular}
		\subsection*{Tasa de compresión: 126 bits / 48 bits = 2.62}	
	\end{center}
	
	%EEEEEEEEEEEEEEEEEEEEEEEEEEEEEEEEEEEEEEEEEEEEEEEEEEEEEe
	\newpage
	\vspace{0.5cm}
	\begin{center}
		\hspace{3cm}\colorbox{yellow}{{\textbf{e.} ccbcbbbcbbb}}\newline
		\subsection*{Total: 11 caracteres x 7 bits = 77 bits}
		\vspace{0.5cm}
		\subsection*{Conteo de caracteres}
		\begin{tabular}{c|c}
			\textbf{Carácter} & \textbf{Frecuencia} \\
			\hline
			b & 7 \\
			c & 4 \\
		\end{tabular}
		
		\vspace{1cm}
		
		\begin{forest}
			for tree={
				parent anchor=south,
				child anchor=north,
				draw,
				l sep=10mm,
				s sep=5mm,
				minimum size=8mm
			}
			[N1:11
			[c, tier=word, edge label={node[midway, left]{0}}]
			[b, tier=word, edge label={node[midway, right]{1}}]
			]
		\end{forest}
		
		
		\subsection*{Codificación resultante}
		\begin{tabular}{c|c|c|c}
			\textbf{Caracter} & \textbf{Frecuencia} & \textbf{Código} & \textbf{Uso de bits}\\
			\hline
			b & 7 & 0 & 7\\
			c & 4 & 1 & 4\\
		\end{tabular}
		\subsection*{Tasa de compresión: 77 bits / 11 bits = 7}	
	\end{center}
	
	%FFFFFFFFFFFFFFFFFFFFFFFFFFFFFFFFFFFFFFFFFFFFFFFFFFFFFFFFFFFFFf	
	\vspace{0.5cm}
	\begin{center}
		\hspace{3cm}\colorbox{yellow}{{\textbf{f.} FFCE FEEF EFED EFFF EDEF EEFE DEFE FFFF FDFF BFFC FDFF FEAE DCDE}}\newline
		\subsection*{Total: 64 caracteres x 7 bits = 448 bits}
		\vspace{0.5cm}
		\subsection*{Conteo de caracteres}
		\begin{tabular}{c|c}
			\textbf{Carácter} & \textbf{Frecuencia} \\
			\hline
			F & 24 \\
			E & 16 \\
			  & 12 \\
			D & 7 \\
			C & 3 \\
			B & 1 \\
			A & 1 \\  
		\end{tabular}
		
		\vspace{1cm}
		
		\begin{forest}
			for tree={
				parent anchor=south,
				child anchor=north,
				draw,
				l sep=10mm,
				s sep=5mm,
				minimum size=8mm
			}
			[N6:64
			[N4:24, edge label={node[midway, left]{0}}
				[N3:12, edge label={node[midway,left]{0}}
					[N2:5, edge label={node[midway,left]{0}}
						[N1:2, edge label={node[midway,left]{0}}
							[A, tier=word, edge label={node[midway, left]{0}}]
							[B, tier=word, edge label={node[midway,right]{1}}]
						]
					[C, tier=word, edge label={node[midway,right]{1}}]	
					]
					[D, tier=word, edge label={node[midway, right]{1}}]	
				]
				[ , tier=word, edge label={node[midway, right]{1}}]
			]	
			[N5:40, edge label={node[midway, right]{1}}
				[E, tier=word, edge label={node[midway, left]{0}}]
				[F, tier=word, edge label={node[midway, right]{1}}]
			]
			]
		\end{forest}
		
		
		\subsection*{Codificación resultante}
		\begin{tabular}{c|c|c|c}
			\textbf{Caracter} & \textbf{Frecuencia} & \textbf{Código} & \textbf{Uso de bits}\\
			\hline
			F & 24 & 11 & 48\\
			E & 16 & 10 & 32\\
			  & 12 & 01 & 24\\
			D & 7  & 001 & 21\\
			C & 3 & 0001 & 12\\
			B & 1 & 00001 & 5\\
			A & 1 & 00000 & 5\\ 
		\end{tabular}
		\subsection*{Tasa de compresión: 448 bits / 147 bits = 3.04}	
	\end{center}
	
	%GGGGGGGGGGGGGGGGGGGGGGGGGGGGGGGGGGGGGGGGGGGGGGGGGGGGGGGGGGGGGG
	\vspace{0.5cm}
	\begin{center}
		\hspace{3cm}\colorbox{yellow}{{\textbf{g.} jonatan imperi}}\newline
		\subsection*{Total: 14 caracteres x 7 bits = 98 bits}
		\vspace{0.5cm}
		\subsection*{Conteo de caracteres}
		\begin{tabular}{c|c}
			\textbf{Carácter} & \textbf{Frecuencia} \\
			\hline
			n & 2 \\
			a & 2 \\
			i & 2 \\
			j & 1 \\
			o & 1 \\
			t & 1 \\
			m & 1 \\  
			p & 1 \\
			e & 1 \\
			r & 1 \\
			  & 1 \\        
		\end{tabular}
		
		\vspace{1cm}
		
		\begin{forest}
			for tree={
				parent anchor=south,
				child anchor=north,
				draw,
				l sep=10mm,
				s sep=5mm,
				minimum size=8mm
			}
			[N10:14
				[N9:8, edge label={node[midway, left]{0}}
					[N6:4, edge label={node[midway,left]{0}}
						[N1:2, edge label={node[midway, left]{0}}
							[ , tier=word, edge label={node[midway, left]{0}}]
							[r, tier=word, edge label={node[midway,right]{1}}]
						]
						[N2:2, edge label={node[midway, right]{1}}
							[e, tier=word, edge label={node[midway, left]{0}}]
							[p, tier=word, edge label={node[midway,right]{1}}]
						]
					]
					[N7:4, edge label={node[midway, right]{1}}
						[N3:2, edge label={node[midway, left]{0}}
							[m, tier=word, edge label={node[midway, left]{0}}]
							[t, tier=word, edge label={node[midway,right]{1}}]
						]
						[N4:2, edge label={node[midway, right]{1}}
							[o, tier=word, edge label={node[midway, left]{0}}]
							[j, tier=word, edge label={node[midway,right]{1}}]
						]
					]	
				]
				[N8:40, edge label={node[midway, right]{1}}
					[N5:4, edge label={node[midway, left]{0}}
						[i, tier=word, edge label={node[midway, left]{0}}]
						[a, tier=word, edge label={node[midway, right]{1}}]
					]
					[n, tier=word, edge label={node[midway, right]{1}}]
				]
			]
		\end{forest}
		
		
		\subsection*{Codificación resultante}
		\begin{tabular}{c|c|c|c}
			\textbf{Caracter} & \textbf{Frecuencia} & \textbf{Código} & \textbf{Uso de bits}\\
			\hline
			n & 2 & 11 & 4\\
			a & 2 & 101 & 6\\
			i & 2 & 100 & 6\\
			j & 1 & 0111 & 4\\
			o & 1 & 0110 & 4\\
			t & 1 & 0101 & 4\\
			m & 1 & 0100 & 4\\  
			p & 1 & 0011 & 4\\
			e & 1 & 0010 & 4\\
			r & 1 & 0001 & 4\\
			  & 1 & 0000 & 4\\
		\end{tabular}
		\subsection*{Tasa de compresión: 98 bits / 48 bits = 2.04}	
	\end{center}
	
	%HHHHHHHHHHHHHHHHHHHHHHHHHHHHHHHHHHHHHHHHHHHHHHHHHHHHHHHHHHHHHHHHHHHHHHHHHHHH
	\vspace{0.5cm}
	\begin{center}
		\hspace{3cm}\colorbox{yellow}{{\textbf{h.} CATCATCATCAT}}\newline
		\subsection*{Total: 12 caracteres x 7 bits = 84 bits}
		\vspace{0.5cm}
		\subsection*{Conteo de caracteres}
		\begin{tabular}{c|c}
			\textbf{Caracter} & \textbf{Frecuencia} \\
			\hline
			C & 4 \\
			A & 4 \\
			T & 4 \\
		\end{tabular}
		
		\vspace{1cm}
		
		\begin{forest}
			for tree={
				parent anchor=south,
				child anchor=north,
				draw,
				l sep=10mm,
				s sep=5mm,
				minimum size=8mm
			}
			[N2:12
			[N1:8, edge label={node[midway, left]{0}}
			[C, tier=word,edge label={node[midway, left]{0}}]
			[A, tier=word,edge label={node[midway, right]{1}}]
			]
			[T, tier=word,edge label={node[midway, right]{1}}]
			]
		\end{forest}
		
		
		\subsection*{Codificación resultante}
		\begin{tabular}{c|c|c|c}
			\textbf{Caracter} & \textbf{Frecuencia} & \textbf{Código} & \textbf{Uso de bits}\\
			\hline
			C & 4 & 00 & 8\\
			A & 4 & 01 & 8\\
			T & 4 & 1 & 4\\
		\end{tabular}
		\subsection*{Tasa de compresión: 84 bits / 20 bits = 4.2}	
	\end{center}
	
	%IIIIIIIIIIIIIIIIIIIIIIIIIIIIIIIIIIIIIIIIIIIIIIIIIIIIIIIIIIIIIIIIIIIII
	\vspace{0.5cm}
	\begin{center}
		\hspace{3cm}\colorbox{yellow}{{\textbf{i.} \emoji{red-heart}\emoji{red-heart}\emoji{red-heart}\emoji{red-heart}\emoji{red-heart}\emoji{red-heart}\emoji{soccer-ball}\emoji{soccer-ball}\emoji{soccer-ball}\emoji{soccer-ball}\emoji{red-triangle-pointed-up}\emoji{red-triangle-pointed-up}\emoji{red-triangle-pointed-up}\emoji{cross-mark}\emoji{cross-mark}\emoji{cross-mark}\emoji{cross-mark}\emoji{cross-mark}\emoji{cross-mark}}}\newline
		\subsection*{Total: 21 caracteres x 32 bits = 672 bits}
		\vspace{0.5cm}
		\subsection*{Conteo de caracteres}
		\begin{tabular}{c|c}
			\textbf{Caracter} & \textbf{Frecuencia} \\
			\hline
			\emoji{red-heart} & 6 \\
			\emoji{cross-mark} & 6 \\
			\emoji{soccer-ball} & 4 \\
			\emoji{red-triangle-pointed-up} & 3 \\
		\end{tabular}
		
		\vspace{0.5cm}
		\begin{forest}
			for tree={
				parent anchor=south,
				child anchor=north,
				draw,
				l sep=10mm,
				s sep=5mm,
				minimum size=8mm
			}
			[N3:19
				[N1:7, edge label={node[midway,left]{0}}
					[{\emoji{red-triangle-pointed-up}}, edge label={node[midway, left]{0}}]
					[{\emoji{soccer-ball}}, edge label={node[midway, right]{1}}]
				]
				[N2:12, edge label={node[midway,right]{1}}
					[{\emoji{cross-mark}}, edge label={node[midway, left]{0}}]
					[{\emoji{red-heart}}, edge label={node[midway, right]{1}}]
				]
			]
		\end{forest}

		\subsection*{Codificación resultante}
		\begin{tabular}{c|c|c|c}
			\textbf{Caracter} & \textbf{Frecuencia} & \textbf{Código} & \textbf{Uso de bits}\\
			\hline
			\emoji{red-heart} & 6 & 11 & 12\\
			\emoji{cross-mark} & 6 & 10 & 12\\
			\emoji{soccer-ball} & 4 & 01 & 8\\
			\emoji{red-triangle-pointed-up} & 3 & 00 & 6\\
		\end{tabular}
		\subsection*{Tasa de compresión: 672 bits / 38 bits = 17.68}	
	\end{center}
	
	%JJJJJJJJJJJJJJJJJJJJJJJJJJJJJJJJJJJJJJJJJJJJJJJJJJJJJJJJJJJJJJJJJJJJJJJJJJJJJJ
	\newpage
	\begin{center}
		\hspace{3cm}\colorbox{yellow}{\textbf{j.}\emoji{crab}\emoji{crab}\emoji{crab}\emoji{crab}\emoji{crab}\emoji{crab}\emoji{crab}\emoji{christmas-tree}\emoji{christmas-tree}\emoji{christmas-tree}\emoji{mouse-face}\emoji{mouse-face}\emoji{dog-face}\emoji{dog-face}\emoji{dog-face}\emoji{dog-face}\emoji{dog-face}} % Solo esto en color
		\newline
		
		\subsection*{Total: 17 caracteres x 32 bits = 544 bits}
		\vspace{0.5cm}
		
		\subsection*{Conteo de caracteres}
		\begin{tabular}{c|c}
			\textbf{Caracter} & \textbf{Frecuencia} \\
			\hline
			\emoji{crab} & 7 \\
			\emoji{dog-face} & 5 \\
			\emoji{christmas-tree} & 3 \\
			\emoji{mouse-face} & 2 \\
		\end{tabular}
		
		\vspace{1cm}
		\begin{forest}
			for tree={
				parent anchor=south,
				child anchor=north,
				draw,
				l sep=10mm,
				s sep=5mm,
				minimum size=8mm
			}
			[N3:17
			[N2:10, edge label={node[midway,left]{0}}
			[N1:5, edge label={node[midway,left]{0}}
			[\emoji{mouse-face}, tier=word, edge label={node[midway, left]{0}}]
			[\emoji{christmas-tree}, tier=word, edge label={node[midway, right]{1}}]
			]
			[\emoji{dog-face}, tier=word, edge label={node[midway, right]{1}}]	
			]
			[\emoji{crab}, tier=word, edge label={node[midway, right]{1}}]
			]
		\end{forest}
		
		\subsection*{Codificación resultante}
		\begin{tabular}{c|c|c|c}
			\textbf{Caracter} & \textbf{Frecuencia} & \textbf{Código} & \textbf{Uso de bits}\\
			\hline
			\emoji{crab} & 7 & 1 & 7\\
			\emoji{dog-face} & 5 & 01 & 10\\
			\emoji{christmas-tree} & 3 & 001 & 9\\
			\emoji{mouse-face} & 2 & 000 & 6\\
		\end{tabular}
		
		\subsection*{Tasa de compresión: 544 bits / 32 bits = 17}	
	\end{center}
	

	2. Halle los bits de paridad en base a los datos transmitidos utilizando el codigo de Hamming.
	\begin{enumerate}
	\item 101010101
	\item 10111001
	\item 0101001
	\item 10101
	\item 10001
	\end{enumerate}
	
	\newpage
	\begin{center}
	\hspace{3cm}\colorbox{yellow}{{\textbf{a.} 101010101}}\newline
	\\Determinar cuantos bits de paridad se necesitan, se prueba con 4
	\[
	2^p \geq p + \text{bits de datos} + 1
	\quad \Rightarrow \quad	2^4 \geq 4 + 9 + 1 \quad \Rightarrow \quad 16 \geq 14
	\]
	
	\begin{table}[h!]
		\centering
		\begin{tabular}{c|cccccccccccccc}
			\toprule
			\textbf{Datos/bits} & \textbf{p1} & \textbf{p2} & \textbf{1} & \textbf{p3} & \textbf{0} & \textbf{1} & \textbf{0} & \textbf{p4} & \textbf{1} & \textbf{0} & \textbf{1} & \textbf{0} & \textbf{1}\\ & \scriptsize001 & \scriptsize010 & \scriptsize011 & \scriptsize100 & \scriptsize101 & \scriptsize110 & \scriptsize111 & \scriptsize1000 & \scriptsize1001 & \scriptsize1010 & \scriptsize1011 & \scriptsize1100 & \scriptsize1011 \\
			\midrule
			\textbf{p1} & \colorbox{pink}{0} & & 1 & & 0 & & 0 & & 1 & & 1 & & 1 \\
			\midrule
			\textbf{p2} &  & \colorbox{pink}{1} & 1 & & & 1 & 0 & & & 0 & 1 & &  \\
			\midrule
			\textbf{p3} &  & & & \colorbox{pink}{0} & 0 & 1 & 0 & & & & & 0 & 1 \\
			\midrule
			\textbf{p4} & & & & & & & & \colorbox{pink}{1} & 1 & 0 & 1 & 0 & 1 \\
			\midrule
			\midrule
			\textbf{Resultado} & 0 & 1 & 1 & 0 & 0 & 1 & 0 & 1 & 1 & 0 & 1 & 0 & 1 \\
			\bottomrule
		\end{tabular}
	\end{table}
	\end{center}	
	%222222222bbbbbbbbbbbbbbbbbbbb
	\begin{center}
		\hspace{3cm}\colorbox{yellow}{{\textbf{b.} 10111001}}\newline
		\\Determinar cuantos bits de paridad se necesitan, se prueba con 4
		\[
		2^p \geq p + \text{bits de datos} + 1
		\quad \Rightarrow \quad	2^4 \geq 4 + 8 + 1 \quad \Rightarrow \quad 16 \geq 13
		\]
		
		\begin{table}[h!]
			\centering
			\begin{tabular}{c|ccccccccccccc}
				\toprule
				\textbf{Datos/bits} & \textbf{p1} & \textbf{p2} & \textbf{1} & \textbf{p3} & \textbf{0} & \textbf{1} & \textbf{1} & \textbf{p4} & \textbf{1} & \textbf{0} & \textbf{0} & \textbf{1} \\ & \scriptsize001 & \scriptsize010 & \scriptsize011 & \scriptsize100 & \scriptsize101 & \scriptsize110 & \scriptsize111 & \scriptsize1000 & \scriptsize1001 & \scriptsize1010 & \scriptsize1011 & \scriptsize1100 \\
				\midrule
				\textbf{p1} & \colorbox{pink}{1} & & 1 & & 0 & & 1 & & 1 & & 0 &\\
				\midrule
				\textbf{p2} &  & \colorbox{pink}{1} & 1 & & & 1 & 1 & & & 0 & 0 &  \\
				\midrule
				\textbf{p3} &  & & & \colorbox{pink}{1} & 0 & 1 & 1 & & & & & 1 \\
				\midrule
				\textbf{p4} & & & & & & & & \colorbox{pink}{0} & 1 & 0 & 0 & 1 \\
				\midrule
				\midrule
				\textbf{Resultado} & 1 & 1 & 1 & 1 & 0 & 1 & 1 & 0 & 1 & 0 & 0 & 1 \\
				\bottomrule
			\end{tabular}
		\end{table}
	\end{center}
	%222222222222222222222cccccccccccccccccccccccccccccccccccccccc
	\begin{center}
		\hspace{3cm}\colorbox{yellow}{{\textbf{c.} 0101001}}\newline
		\\Determinar cuantos bits de paridad se necesitan, se prueba con 4
		\[
		2^p \geq p + \text{bits de datos} + 1
		\quad \Rightarrow \quad	2^4 \geq 4 + 7 + 1 \quad \Rightarrow \quad 16 \geq 11
		\]
		
		\begin{table}[h!]
			\centering
			\begin{tabular}{c|ccccccccccccc}
				\toprule
				\textbf{Datos/bits} & \textbf{p1} & \textbf{p2} & \textbf{0} & \textbf{p3} & \textbf{1} & \textbf{0} & \textbf{1} & \textbf{p4} & \textbf{0} & \textbf{0} & \textbf{1} \\ & \scriptsize001 & \scriptsize010 & \scriptsize011 & \scriptsize100 & \scriptsize101 & \scriptsize110 & \scriptsize111 & \scriptsize1000 & \scriptsize1001 & \scriptsize1010 & \scriptsize1011 \\
				\midrule
				\textbf{p1} & \colorbox{pink}{1} & & 0 & & 1 & & 1 & & 0 & & 0 \\
				\midrule
				\textbf{p2} &  & \colorbox{pink}{0} & 0 & & & 0 & 0 & & & 0 &  \\
				\midrule
				\textbf{p3} & & & & \colorbox{pink}{1} & 1 & 0 & 1 & & & & 1 \\
				\midrule
				\textbf{p4} & & & & & & & & \colorbox{pink}{1} & 0 & 0 & 1 \\
				\midrule
				\midrule
				\textbf{Resultado} & 1 & 0 & 0 & 1 & 1 & 0 & 1 & 1 & 0 & 0 & 1\\
				\bottomrule
			\end{tabular}
		\end{table}
	\end{center}
	%2222222222222222222222ddddddddddddddddddddddddddddddddddddddddddddddddddddd
	\newpage
	\begin{center}
		\hspace{3cm}\colorbox{yellow}{{\textbf{d.} 10101}}\newline
		\\Determinar cuantos bits de paridad se necesitan, se prueba con 4
		\[
		2^p \geq p + \text{bits de datos} + 1
		\quad \Rightarrow \quad	2^4 \geq 4 + 5 + 1 \quad \Rightarrow \quad 16 \geq 10
		\]
		
		\begin{table}[h!]
			\centering
			\begin{tabular}{c|ccccccccccccc}
				\toprule
				\textbf{Datos/bits} & \textbf{p1} & \textbf{p2} & \textbf{1} & \textbf{p3} & \textbf{0} & \textbf{1} & \textbf{0} & \textbf{p4} & \textbf{1}\\ & \scriptsize001 & \scriptsize010 & \scriptsize011 & \scriptsize100 & \scriptsize101 & \scriptsize110 & \scriptsize111 & \scriptsize1000 & \scriptsize1001\\
				\midrule
				\textbf{p1} & \colorbox{pink}{0} & & 1 & & 0 & & 0 & & 1\\
				\midrule
				\textbf{p2} &  & \colorbox{pink}{0} & 1 & & & 1 & 0 & & \\
				\midrule
				\textbf{p3} & & & & \colorbox{pink}{1} & 0 & 1 & 0 & & \\
				\midrule
				\textbf{p4} & & & & & & & & \colorbox{pink}{1} & 1\\
				\midrule
				\midrule
				\textbf{Resultado} & 0 & 0 & 1 & 1 & 0 & 1 & 0 & 1 & 1\\
				\bottomrule
			\end{tabular}
		\end{table}
	\end{center}
	%222222222222222222222222222222222222eeeeeeeeeeeeeeeeeeeeeeeeeeeeeeeeeeeeee
	\begin{center}
		\hspace{3cm}\colorbox{yellow}{{\textbf{e.} 10001}}\newline
		\\Determinar cuantos bits de paridad se necesitan, se prueba con 4
		\[
		2^p \geq p + \text{bits de datos} + 1
		\quad \Rightarrow \quad	2^4 \geq 4 + 5 + 1 \quad \Rightarrow \quad 16 \geq 10
		\]
		
		\begin{table}[h!]
			\centering
			\begin{tabular}{c|ccccccccccccc}
				\toprule
				\textbf{Datos/bits} & \textbf{p1} & \textbf{p2} & \textbf{1} & \textbf{p3} & \textbf{0} & \textbf{0} & \textbf{0} & \textbf{p4} & \textbf{1}\\ & \scriptsize001 & \scriptsize010 & \scriptsize011 & \scriptsize100 & \scriptsize101 & \scriptsize110 & \scriptsize111 & \scriptsize1000 & \scriptsize1001\\
				\midrule
				\textbf{p1} & \colorbox{pink}{0} & & 1 & & 0 & & 0 & & 1\\
				\midrule
				\textbf{p2} &  & \colorbox{pink}{1} & 1 & & & 0 & 0 & & \\
				\midrule
				\textbf{p3} & & & & \colorbox{pink}{0} & 0 & 0 & 0 & & \\
				\midrule
				\textbf{p4} & & & & & & & & \colorbox{pink}{1} & 1\\
				\midrule
				\midrule
				\textbf{Resultado} & 0 & 1 & 1 & 0 & 0 & 0 & 0 & 1 & 1\\
				\bottomrule
			\end{tabular}
		\end{table}
	\end{center}
	%3333333333333333333333333333333AAAAAAAAAAAAAAAAAAAAAA
	\vspace{1cm}
	\newpage
	3. Determina los bits de paridad y forma el mensaje codificado utilizando el codigo de
	Hamming.
	\begin{enumerate}
	\item Mensaje original: 1101
	\item Mensaje original: 1010
	\item Mensaje original: 0110
	\item Mensaje original: 1110
	\item Mensaje original: 0101
	\end{enumerate}
	
	%33333333333333333333333AAAAAAAAAAAAAAAAAAAAAAAAAAAAAAAAAAAAA
	\begin{center}
		\hspace{3cm}\colorbox{yellow}{{\textbf{a.} 1101}}\newline
		\\Determinar cuantos bits de paridad se necesitan, se prueba con 4
		\[
		2^p \geq p + \text{bits de datos} + 1
		\quad \Rightarrow \quad	2^3 \geq 3 + 4 + 1 \quad \Rightarrow \quad 8 \geq 8
		\]
		
		\begin{table}[h!]
			\centering
			\begin{tabular}{c|ccccccc}
				\toprule
				\textbf{Datos/bits} & \textbf{p1} & \textbf{p2} & \textbf{1} & \textbf{p3} & \textbf{1} & \textbf{0} & \textbf{1}\\ & \scriptsize001 & \scriptsize010 & \scriptsize011 & \scriptsize100 & \scriptsize101 & \scriptsize110 & \scriptsize111\\
				\midrule
				\textbf{p1} & \colorbox{pink}{1} & & 1 & & 1 & & 1\\
				\midrule
				\textbf{p2} &  & \colorbox{pink}{0} & 1 & & 0 & 1  \\
				\midrule
				\textbf{p3} & & & & \colorbox{pink}{0} & 1 & 0 & 1 \\
				\midrule
				\textbf{Resultado} & 1 & 0 & 1 & 0 & 1 & 0 & 1\\
				\bottomrule
			\end{tabular}
		\end{table}
	\end{center}
	
	%33333333333333333333333333333333333333333333333BBBBBBBBBBBBBBBBBBBBBBBBBBBBBBBBBBBBB
		\begin{center}
		\hspace{3cm}\colorbox{yellow}{{\textbf{b.} 1010}}\newline
		\\Determinar cuantos bits de paridad se necesitan, se prueba con 4
		\[
		2^p \geq p + \text{bits de datos} + 1
		\quad \Rightarrow \quad	2^3 \geq 3 + 4 + 1 \quad \Rightarrow \quad 8 \geq 8
		\]
		
		\begin{table}[h!]
			\centering
			\begin{tabular}{c|ccccccc}
				\toprule
				\textbf{Datos/bits} & \textbf{p1} & \textbf{p2} & \textbf{1} & \textbf{p3} & \textbf{0} & \textbf{1} & \textbf{0}\\ & \scriptsize001 & \scriptsize010 & \scriptsize011 & \scriptsize100 & \scriptsize101 & \scriptsize110 & \scriptsize111\\
				\midrule
				\textbf{p1} & \colorbox{pink}{1} & & 1 & & 0 & & 0\\
				\midrule
				\textbf{p2} &  & \colorbox{pink}{0} & 1 & & 1 & 0  \\
				\midrule
				\textbf{p3} & & & & \colorbox{pink}{1} & 0 & 1 & 0 \\
				\midrule
				\textbf{Resultado} & 1 & 0 & 1 & 1 & 0 & 1 & 0\\
				\bottomrule
			\end{tabular}
		\end{table}
	\end{center}
	%3333333333333333333333CCCCCCCCCCCCCCCCCCCCCCCCCCCCCCCCCC
		\begin{center}
		\hspace{3cm}\colorbox{yellow}{{\textbf{c.} 0110}}\newline
		\\Determinar cuantos bits de paridad se necesitan, se prueba con 4
		\[
		2^p \geq p + \text{bits de datos} + 1
		\quad \Rightarrow \quad	2^3 \geq 3 + 4 + 1 \quad \Rightarrow \quad 8 \geq 8
		\]
		
		\begin{table}[h!]
			\centering
			\begin{tabular}{c|ccccccc}
				\toprule
				\textbf{Datos/bits} & \textbf{p1} & \textbf{p2} & \textbf{1} & \textbf{p3} & \textbf{0} & \textbf{1} & \textbf{0}\\ & \scriptsize001 & \scriptsize010 & \scriptsize011 & \scriptsize100 & \scriptsize101 & \scriptsize110 & \scriptsize111\\
				\midrule
				\textbf{p1} & \colorbox{pink}{1} & & 0 & & 1 & & 0\\
				\midrule
				\textbf{p2} &  & \colorbox{pink}{1} & 0 & & 1 & 0  \\
				\midrule
				\textbf{p3} & & & & \colorbox{pink}{0} & 1 & 1 & 0 \\
				\midrule
				\textbf{Resultado} & 1 & 1 & 0 & 0 & 1 & 1 & 0\\
				\bottomrule
			\end{tabular}
		\end{table}
	\end{center}
	%3333333333333333333333333DDDDDDDDDDDDDDDDDDDDDDDDDDDDDDDDDDDDDDDDDDDDDddd
	\begin{center}
		\hspace{3cm}\colorbox{yellow}{{\textbf{d.} 1110}}\newline
		\\Determinar cuantos bits de paridad se necesitan, se prueba con 4
		\[
		2^p \geq p + \text{bits de datos} + 1
		\quad \Rightarrow \quad	2^3 \geq 3 + 4 + 1 \quad \Rightarrow \quad 8 \geq 8
		\]
		
		\begin{table}[h!]
			\centering
			\begin{tabular}{c|ccccccc}
				\toprule
				\textbf{Datos/bits} & \textbf{p1} & \textbf{p2} & \textbf{1} & \textbf{p3} & \textbf{1} & \textbf{1} & \textbf{0}\\ & \scriptsize001 & \scriptsize010 & \scriptsize011 & \scriptsize100 & \scriptsize101 & \scriptsize110 & \scriptsize111\\
				\midrule
				\textbf{p1} & \colorbox{pink}{0} & & 1 & & 1 & & 0\\
				\midrule
				\textbf{p2} &  & \colorbox{pink}{0} & 1 & & 1 & 0  \\
				\midrule
				\textbf{p3} & & & & \colorbox{pink}{0} & 1 & 1 & 0 \\
				\midrule
				\textbf{Resultado} & 0 & 0 & 1 & 0 & 1 & 1 & 0\\
				\bottomrule
			\end{tabular}
		\end{table}
	\end{center}
	%33333333333333333333333333333333333333333333EEEEEEEEEEEEEEEEEEEEEEEEEEEEEEEEEE
	\begin{center}
		\hspace{3cm}\colorbox{yellow}{{\textbf{e.} 0101}}\newline
		\\Determinar cuantos bits de paridad se necesitan, se prueba con 5
		\[
		2^p \geq p + \text{bits de datos} + 1
		\quad \Rightarrow \quad	2^5 \geq 3 + 4 + 1 \quad \Rightarrow \quad 8 \geq 8
		\]
		
		\begin{table}[h!]
			\centering
			\begin{tabular}{c|ccccccc}
				\toprule
				\textbf{Datos/bits} & \textbf{p1} & \textbf{p2} & \textbf{0} & \textbf{p3} & \textbf{1} & \textbf{0} & \textbf{1}\\ & \scriptsize001 & \scriptsize010 & \scriptsize011 & \scriptsize100 & \scriptsize101 & \scriptsize110 & \scriptsize111\\
				\midrule
				\textbf{p1} & \colorbox{pink}{1} & & 0 & & 1 & & 0\\
				\midrule
				\textbf{p2} &  & \colorbox{pink}{1} & 0 & & 0 & 1 \\
				\midrule
				\textbf{p3} & & & & \colorbox{pink}{0} & 1 & 0 & 1 \\
				\midrule
				\textbf{Resultado} & 0 & 1 & 0 & 0 & 1 & 0 & 1\\
				\bottomrule
			\end{tabular}
		\end{table}
		
		
	\end{center}

	4. El dato recibido por un MODEM y protegido mediante código Hamming es el siguiente:011100110101010110 Se pide:
	\begin{enumerate}
	 \item Calcular si el número recibido es correcto.
	 \item Si no es correcto, corregir el número.
	\end{enumerate} 
	
	\begin{center}
		\hspace{3cm}\colorbox{yellow}{{\textbf{dato recibido:} 011100110101010110}}\newline
		\\Determinar cuantos bits de paridad se necesitan, se prueba con 4
		\[
		2^p \geq p + \text{bits de datos} + 1
		\quad \Rightarrow \quad	2^5 \geq 5 + 14 + 1 \quad \Rightarrow \quad 32 \geq 20
		\]
	\end{center}
	
	
	\begin{table}[h!]
		\footnotesize % o \scriptsize para reducir un poco
		\setlength{\tabcolsep}{4pt} % reduce separación entre columnas
		\begin{tabularx}{\textwidth}{c|*{18}{>{\centering\arraybackslash}X}}
			\toprule
			\textbf{Datos/bits} & \textbf{0} & \textbf{1} & \textbf{1} & \textbf{1} & \textbf{0} & \textbf{0} & \textbf{1} & \textbf{1} & \textbf{0} & \textbf{1} & \textbf{0} & \textbf{1} & \textbf{0} & \textbf{1} & \textbf{0} & \textbf{1} & \textbf{1} & \textbf{0} \\
			& \scriptsize001 & \scriptsize010 & \scriptsize011 & \scriptsize100 & \scriptsize101 & \scriptsize110 & \scriptsize111 & \scriptsize1000 & \scriptsize1001 & \scriptsize1010 & \scriptsize1011 & \scriptsize1100 & \scriptsize1101 & \scriptsize1110 & \scriptsize1111 & \scriptsize10000 & \scriptsize10001 & \scriptsize10010\\
			\midrule
			\textbf{} & p1 & p2 & & p3 & & & & p4 & & & & & & & & p5 & &  \\
			\midrule
			\textbf{p1} & \colorbox{red}{1} & & 1 & & 0 & & 1 & & 0 & & 0 & & 0 & & 0 & & 1 &  \\
			\midrule
			\textbf{p2} &  & \colorbox{red}{0} & 1 & & & 0 & 1 & & & 1 & 0 & & & 1 & 0 & & & 0 \\
			\midrule
			\textbf{p3} & & & & \colorbox{green}{1} & 0 & 0 & 1 & & & & & 1 & 0 & 1 & 0 & & & \\
			\midrule
			\textbf{p4} & & & & & & & & \colorbox{green}{1} & 0 & 1 & 0 & 1 & 0 & 1 & 0 & & & \\
			\midrule
			\textbf{p5} & & & & & & & & &  &  &  &  &  &  &  & \colorbox{green}{1} & 1 & 0\\
			\midrule
			\textbf{Resultado} & & & & & & & & &  &  &  &  &  &  &  &  &  & \\
			\bottomrule
		\end{tabularx}
	\end{table}
	
	\begin{center}
		\vspace{0.5cm}
		\Xor{p1~p2~p3~p4}{0~~1~~1~~1}{1~~0~~1~~1}[1~~1~~0~~0]\\
	\end{center}	
	\vspace{0.5cm}
	Leyendo el resultado en orden inverso nos queda 0011 que equivale al número 3, ese es el bit recibido con error.\\
	Dato recibido: 01\colorbox{red}{1}100110101010110\\
	Dato corregido 10\colorbox{green}{0}100110101010110\\
		
	\vspace{1cm}
	5. Se tiene un computador conectado a una red telefónica por la que llegan números en
	coma flotante relativos a las temperaturas en una cámara frigorífica de carnes. Los
	datos vienen protegidos mediante código Hamming. A nuestro terminal ha llegado
	el dato siguiente: 0 1 1 1 0 0 1 1 0 1 0 1 0 1 0 1 1 0
	Se pide: Verificar si el número llega correctamente y si fuese necesario realizar la
	corrección correspondiente.

	\begin{center}
		\hspace{3cm}\colorbox{yellow}{{\textbf{dato recibido:} 011100110101010110}}\newline
		\\Determinar cuantos bits de paridad se necesitan, se prueba con 4
		\[
		2^p \geq p + \text{bits de datos} + 1
		\quad \Rightarrow \quad	2^5 \geq 5 + 14 + 1 \quad \Rightarrow \quad 32 \geq 20
		\]
	\end{center}
	
	
	\begin{table}[h!]
		\footnotesize % o \scriptsize para reducir un poco
		\setlength{\tabcolsep}{4pt} % reduce separación entre columnas
		\begin{tabularx}{\textwidth}{c|*{18}{>{\centering\arraybackslash}X}}
			\toprule
			\textbf{Datos/bits} & \textbf{0} & \textbf{1} & \textbf{1} & \textbf{1} & \textbf{0} & \textbf{0} & \textbf{1} & \textbf{1} & \textbf{0} & \textbf{1} & \textbf{0} & \textbf{1} & \textbf{0} & \textbf{1} & \textbf{0} & \textbf{1} & \textbf{1} & \textbf{0} \\
			& \scriptsize001 & \scriptsize010 & \scriptsize011 & \scriptsize100 & \scriptsize101 & \scriptsize110 & \scriptsize111 & \scriptsize1000 & \scriptsize1001 & \scriptsize1010 & \scriptsize1011 & \scriptsize1100 & \scriptsize1101 & \scriptsize1110 & \scriptsize1111 & \scriptsize10000 & \scriptsize10001 & \scriptsize10010\\
			\midrule
			\textbf{} & p1 & p2 & & p3 & & & & p4 & & & & & & & & p5 & &  \\
			\midrule
			\textbf{p1} & \colorbox{red}{1} & & 1 & & 0 & & 1 & & 0 & & 0 & & 0 & & 0 & & 1 &  \\
			\midrule
			\textbf{p2} &  & \colorbox{red}{0} & 1 & & & 0 & 1 & & & 1 & 0 & & & 1 & 0 & & & 0 \\
			\midrule
			\textbf{p3} & & & & \colorbox{green}{1} & 0 & 0 & 1 & & & & & 1 & 0 & 1 & 0 & & & \\
			\midrule
			\textbf{p4} & & & & & & & & \colorbox{green}{1} & 0 & 1 & 0 & 1 & 0 & 1 & 0 & & & \\
			\midrule
			\textbf{p5} & & & & & & & & &  &  &  &  &  &  &  & \colorbox{green}{1} & 1 & 0\\
			\midrule
			\textbf{Resultado} & & & & & & & & &  &  &  &  &  &  &  &  &  & \\
			\bottomrule
		\end{tabularx}
	\end{table}
	
		\begin{center}
		\vspace{0.5cm}
		\Xor{p1~p2~p3~p4}{0~~1~~1~~1}{1~~0~~1~~1}[1~~1~~0~~0]\\
	\end{center}	
	\vspace{0.5cm}
	Leyendo el resultado en orden inverso nos queda 0011 que equivale al número 3, ese es el bit recibido con error.\\
	Dato recibido: 01\colorbox{red}{1}100110101010110\\
	Dato corregido 10\colorbox{green}{0}100110101010110\\
	\newpage
	
	6. Dado el siguiente árbol binario, obtener el código de Hauffman una de las variantes posibles de string.	
	\begin{center}
	\begin{forest}
		for tree={
			parent anchor=south,
			child anchor=north,
			draw,
			l sep=8mm,
			s sep=5mm,
			minimum size=8mm,
			fill=cyan!30, % color de fondo para todos los nodos
			fill=cyan!30, % color de fondo para todos los nodos
			if n children=0{ % Si el nodo es hoja
				tier=word,
			}{
				shape=circle, % si NO es hoja, lo hace circular
			},
		}
		[15
			[C:6, tier=word, edge label={node[midway, left]{0}}
			]
			[9, edge label={node[midway,right]{1}}
				[4, edge label={node[midway,left]{0}}
					[B:1, tier=word, edge label={node[midway, left]{0}}]
					[D:3, tier=word, edge label={node[midway, right]{1}}]
				]	
					[A:5, tier=word, edge label={node[midway, right]{1}}]		
				]
		]			
	\end{forest}
	\end{center}
	\vspace{0.5cm}
	\begin{minipage}{0.5\textwidth}
	\begin{tabular}{c|c|c|c}
		\textbf{Caracter} & \textbf{Frecuencia} & \textbf{Código} \\
		\hline
		B & 1 & 100 \\
		D & 3 & 101 \\
		A & 5 & 11  \\
		C & 6 & 0   \\
	\end{tabular}
	\end{minipage}
	\begin{minipage}{0.5\textwidth}
	Posible String: ABCDACDACDACACC
	\end{minipage}
	\vspace{1cm}
	
	7. Construya el arbol de Hauffman con las siguientes frecuencias:
	\vspace{0.5cm}
	\begin{center}
	\noindent
	\begin{minipage}{0.3\textwidth}
		\begin{tabular}{c|c}
			A & 15 \\
			B & 6  \\
			C & 7  \\
		\end{tabular}
	\end{minipage}
	\hfill
	\begin{minipage}{0.3\textwidth}
		\begin{tabular}{c|c}
			D & 12 \\
			E & 25  \\
			F & 4	 \\
		\end{tabular}
	\end{minipage}
	\hfill
	\begin{minipage}{0.3\textwidth}
		\begin{tabular}{c|c}
			G & 6 \\
			H & 3  \\
			I & 15  \\
		\end{tabular}
	\end{minipage}
	\end{center}
	\vspace{0.5cm}
	\noindent
	\begin{minipage}{0.32\textwidth}
		\begin{forest}
			for tree={
				parent anchor=south,
				child anchor=north,
				draw,
				l sep=10mm,
				s sep=5mm,
				minimum size=8mm,
				scale=0.7,
				transform shape
			}
			[N2:28
			[N1:13, edge label={node[midway, left]{0}}
			[B:6, tier=word, edge label={node[midway, left]{0}}]
			[C:7, tier=word, edge label={node[midway, right]{1}}]
			]
			[A:15, tier=word, edge label={node[midway, right]{1}}]
			]
		\end{forest}
	\end{minipage}%
	\hfill
	\begin{minipage}{0.32\textwidth}
		\begin{forest}
			for tree={
				parent anchor=south,
				child anchor=north,
				draw,
				l sep=10mm,
				s sep=5mm,
				minimum size=8mm,
				scale=0.7,
				transform shape
			}
			[N2:24
			[N1:9, edge label={node[midway, left]{0}}
			[H:3, tier=word, edge label={node[midway, left]{0}}]
			[G:6, tier=word, edge label={node[midway, right]{1}}]
			]
			[I:15, tier=word, edge label={node[midway, right]{1}}]
			]
		\end{forest}
	\end{minipage}%
	\hfill
	\begin{minipage}{0.32\textwidth}
		\begin{forest}
			for tree={
				parent anchor=south,
				child anchor=north,
				draw,
				l sep=10mm,
				s sep=5mm,
				minimum size=8mm,
				scale=0.7,
				transform shape
			}
			[N2:41
			[N1:16, edge label={node[midway, left]{0}}
			[F:4, tier=word, edge label={node[midway, left]{0}}]
			[D:12, tier=word, edge label={node[midway, right]{1}}]
			]
			[E:25, tier=word, edge label={node[midway, right]{1}}]
			]
		\end{forest}
	\end{minipage}
	\vspace{1.5cm}
	
	8. Que debe hacer el receptor si recibe cada uno de estos codigos de Hamming?
	\begin{enumerate}
	\item 0 1 1 1 1 1 0 
	\item 1 1 1 0 0 0 0 
	\item 0 1 0 1 1 1 0 
	\item 0 1 1 1 0 1 1 
	\end{enumerate}
	%8888888AAAAAAAAAAAAAAAAAAAA
	\newpage
	\begin{center}
		\hspace{3cm}\colorbox{yellow}{{\textbf{a.} 0111110}}\newline
		\\Cálculo de control a bits de paridad
		
		\begin{table}[h!]
			\centering
			\begin{tabular}{c|ccccccc}
				\toprule
				\textbf{Datos/bits} & \textbf{0} & \textbf{1} & \textbf{1} & \textbf{1} & \textbf{1} & \textbf{1} & \textbf{0}\\ & \scriptsize001 & \scriptsize010 & \scriptsize011 & \scriptsize100 & \scriptsize101 & \scriptsize110 & \scriptsize111\\
				\midrule
				\textbf{} & p1 & p2 & & p3 & & & \\
				\midrule
				\textbf{p1} & \colorbox{green}{0} & & 1 & & 1 & & 0\\
				\midrule
				\textbf{p2} &  & \colorbox{red}{0} & 1 & & 1 & 0  \\
				\midrule
				\textbf{p3} & & & & \colorbox{red}{0} & 1 & 1 & 0 \\
				\midrule
				\textbf{Resultado} & 0 & 0 & 1 & 0 & 1 & 1 & 0\\
				\bottomrule
			\end{tabular}
		\end{table}
	\end{center}

		\begin{center}
		Haciendo un Xor entre los bits de paridad dados, y los nuevos calculados
		\vspace{0.1cm}
		\Xor{p1~p2~p3}{0~~1~~1}{0~~0~~0}[0~~1~~1]\\
		
		\vspace{0.5cm}
		Se lee el resultado en orden inverso nos queda 011 que equivale al número 6, ese es el bit recibido con error.
		\end{center}
	\vspace{0.3cm}
	Dato recibido: 01111\colorbox{red}{1}0\\\\
	Dato corregido 01111\colorbox{green}{0}0\\
	
	
	%8888888888888888888888888888888888888BBBBBBBBBBBBBBBBBBBBBBBBBBBBBBBBBBBBBBBBB
	\begin{center}
		\hspace{3cm}\colorbox{yellow}{{\textbf{b.} 1110000}}\newline
		\\Cálculo de control a bits de paridad
		
		\begin{table}[h!]
			\centering
			\begin{tabular}{c|ccccccc}
				\toprule
				\textbf{Datos/bits} & \textbf{1} & \textbf{1} & \textbf{1} & \textbf{0} & \textbf{0} & \textbf{0} & \textbf{0}\\ & \scriptsize001 & \scriptsize010 & \scriptsize011 & \scriptsize100 & \scriptsize101 & \scriptsize110 & \scriptsize111\\
				\midrule
				\textbf{} & p1 & p2 & & p3 & & & \\
				\midrule
				\textbf{p1} & \colorbox{green}{1} & & 1 & & 0 & & 0\\
				\midrule
				\textbf{p2} &  & \colorbox{green}{1} & 1 & & 0 & 0  \\
				\midrule
				\textbf{p3} & & & & \colorbox{green}{0} & 0 & 0 & 0 \\
				\midrule
				\textbf{Resultado} & 1 & 1 & 1 & 0 & 0 & 0 & 0\\
				\bottomrule
			\end{tabular}
		\end{table}
	\end{center}
	
	\begin{center}
		El código recibido no contiene error 
	\end{center}

	%88888888888888888888888888888888cccccccccccccccccccccccccccccccccccccccccccccccccccccccc
	\newpage
	\begin{center}
		\hspace{3cm}\colorbox{yellow}{{\textbf{c.} 0101110}}\newline
		\\Cálculo de control a bits de paridad
		
		\begin{table}[h!]
			\centering
			\begin{tabular}{c|ccccccc}
				\toprule
				\textbf{Datos/bits} & \textbf{0} & \textbf{1} & \textbf{0} & \textbf{1} & \textbf{1} & \textbf{1} & \textbf{0}\\ & \scriptsize001 & \scriptsize010 & \scriptsize011 & \scriptsize100 & \scriptsize101 & \scriptsize110 & \scriptsize111\\
				\midrule
				\textbf{} & p1 & p2 & & p3 & & & \\
				\midrule
				\textbf{p1} & \colorbox{red}{1} & & 0 & & 1 & & 0\\
				\midrule
				\textbf{p2} &  & \colorbox{green}{1} & 0 & & 1 & 0  \\
				\midrule
				\textbf{p3} & & & & \colorbox{red}{0} & 1 & 1 & 0 \\
				\midrule
				\textbf{Resultado} & 1 & 1 & 0 & 0 & 1 & 1 & 0\\
				\bottomrule
			\end{tabular}
		\end{table}
	\end{center}
	
	\begin{center}
		Haciendo un Xor entre los bits de paridad dados, y los nuevos calculados
		\vspace{0.1cm}
		\Xor{p1~p2~p3}{0~~1~~1}{1~~1~~0}[1~~0~~1]\\
		
		\vspace{0.5cm}
		Se lee el resultado en orden inverso nos queda 101 que equivale al número 5, ese es el bit recibido con error.
	\end{center}
	\vspace{0.3cm}
	Dato recibido: 0101\colorbox{red}{1}10\\\\
	Dato corregido 0101\colorbox{green}{0}10\\
	
	%888888888888888888888888888DDDDDDDDDDDDDDDDDDDDDDDDDDDDDDDDDDD
	\begin{center}
		\hspace{3cm}\colorbox{yellow}{{\textbf{d.} 0111011}}\newline
		\\Cálculo de control a bits de paridad
		
		\begin{table}[h!]
			\centering
			\begin{tabular}{c|ccccccc}
				\toprule
				\textbf{Datos/bits} & \textbf{0} & \textbf{1} & \textbf{1} & \textbf{1} & \textbf{0} & \textbf{1} & \textbf{1}\\ & \scriptsize001 & \scriptsize010 & \scriptsize011 & \scriptsize100 & \scriptsize101 & \scriptsize110 & \scriptsize111\\
				\midrule
				\textbf{} & p1 & p2 & & p3 & & & \\
				\midrule
				\textbf{p1} & \colorbox{green}{0} & & 1 & & 0 & & 1\\
				\midrule
				\textbf{p2} &  & \colorbox{green}{1} & 1 & & 1 & 1  \\
				\midrule
				\textbf{p3} & & & & \colorbox{red}{0} & 0 & 1 & 1 \\
				\midrule
				\textbf{Resultado} & 0 & 1 & 1 & 0 & 0 & 1 & 1\\
				\bottomrule
			\end{tabular}
		\end{table}
	\end{center}
	
	\begin{center}
		Haciendo un Xor entre los bits de paridad dados, y los nuevos calculados
		\vspace{0.1cm}
		\Xor{p1~p2~p3}{0~~1~~1}{0~~1~~0}[0~~0~~1]\\
		
		\vspace{0.5cm}
		Se lee el resultado en orden inverso nos queda 100 que equivale al número 4, ese es el bit recibido con error.
	\end{center}
	\vspace{0.3cm}
	Dato recibido: 011\colorbox{red}{1}011\\\\
	Dato corregido 011\colorbox{green}{0}011\\
	
	
	
	
	
	
	
	
	
	
	
	
	
	
	
	
	
	
	
	
	
	
	
	
	
	
	
	
	
	
	
	
	
	
	
	
\end{document}
