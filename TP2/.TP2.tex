\documentclass[a4paper,12pt]{article}
\usepackage[utf8]{inputenc}
\usepackage{amsmath}
\usepackage{listings}
\usepackage{fancyhdr}
\usepackage{caption}
\usepackage{graphicx}
\usepackage{xcolor}
\usepackage{xparse}
\usepackage{enumitem}
\usepackage{float}
\usepackage[a4paper, top=2.5cm, bottom=2cm, left=2cm, right=2cm]{geometry}

\pagestyle{fancy}
\fancyhf{}
\rhead{INSPT - UTN}
\lhead{Jonatan Imperi}
\cfoot{\thepage}


\NewDocumentCommand{\Suma}{m m m m O{}}{
	\begin{minipage}{0.28\linewidth}
		\textbf{\hspace{1cm} #4)}\\[-0.5em]
		\[
		\begin{array}{r}
			\textcolor{red}{\scriptsize \texttt{#5}} \\[-0.2em]
			\texttt{\Large #1} \\
			\text{+\hspace{2.0cm}} \\
			\texttt{\Large #2} \\[0.1em]
			\hline \\[-0.7em]
			\texttt{\Large #3}
		\end{array}
		\]
	\end{minipage}
}

\NewDocumentCommand{\Resta}{m m m m O{}}{
	\begin{minipage}{0.28\linewidth}
		\textbf{\hspace{1cm} #4)}\\[-0.5em]
		\[
		\begin{array}{r}
			\textcolor{blue}{\scriptsize \texttt{#5}} \\ [-0.2em]
			\texttt{\Large #1} \\
			\text{-\hspace{2.0cm}}\\ 
			\texttt{\Large #2} \\[0.1em] 
			\hline \\ [-0.7em]
			\texttt{\Large #3}
		\end{array}
		\]
	\end{minipage}
}

\NewDocumentCommand{\Xor}{m m m o}{
	\begin{minipage}{0.22\linewidth}
		\IfValueT{#4}{\textbf{#4}\\[-0.5em]}
		\[
		\begin{array}{r}
			\texttt{#1} \\
			\text{Xor\hspace{1.0cm}} \\
			\texttt{#2} \\
			\hline
			\texttt{#3}
		\end{array}
		\]
	\end{minipage}
}


\begin{document}
	
	\begin{center}
		
		\LARGE \textbf{Sistemas de computación 1} \\[0.5cm]
		\LARGE Trabajo práctico n° 2 \\
	\end{center}
	
	\section*{Aritmética binaria, octal y hexadecimal}
	1. Resolver las siguientes operaciones en BINARIO (Resultado y operaciones deben estar en el desarrollo).
	\renewcommand{\theenumi}{\alph{enumi}}
	\begin{enumerate}
		\item  10110 + 101001
		\item  100111 + 1011
		\item  111001 + 11011
		\item  100010 - 1011
		\item  111000 - 100111
		\item  101101 - 1111 
	\end{enumerate}

	\subsection*{Método de resolución}
	Para llevar a cabo las operaciones entre números binarios se alinearon a partir de la derecha coincidiendo sus correspondientes posiciones. La suma se efectúa de igual forma que en los números decimales, pero en el caso de operar 1+1 (que es 2 en decimal y 10 en binario) se genera un acarreo (marcados en rojo) quedando 0 bajo de la línea de igualdad y "llevando" 1 para ser sumado en la siguiente posición (bit de mayor peso). Si el acarreo llevara un 1 y se tendrían que sumar otros 1 (siendo 3 el resultado en decimal y 11 en binario) queda 1 bajo la línea de igualdad y 1 pasa al bit de mayor peso.
	Para la resta también se opera de forma similar a la decimal y cuando no puede hacerse como ser el caso de 0-1 "se le pide" a la posición de la izquierda (bit de mayor peso), y si este llegara a ser también 0 se llega hasta el próximo 1. En estos sucesivos préstamos siempre se le van restando 1 de forma que una vez que se llega a la cifra que primero pidió el préstamo terminan quedando 10 (2 en decimal) mientras que las intermedias quedan en 1 (en azul quedan marcadas como termina cada posición).\\
		
	\begin{center}
		\begin{tabular}{ccc}
			\Suma{10110}{101001}{111111}{a}&
			\Suma{100111}{1011}{110010}{b} [~1~1~1~1~~~]&
			\Suma{111001}{11011}{1010100}{c} [1~1~~~1~1~~~]\\ \\ 
			\Resta{100010}{1011}{10111}{d} [1~1~1 10 10] &
			\Resta{111000}{100111}{10001}{e} [0 1 1 10] &
			\Resta{101101}{1111}{11110}{f} [1|10|10|10~~]\\
		\end{tabular}
	\end{center}
	
	2. Resolver las siguientes operaciones en OCTAL (resultado y operaciones deben estar en el desarrollo).
	\begin{enumerate}
		\item  456 + 123
		\item  507 + 265
		\item  413 - 256
		\item  602 - 375
		\item  530 - 164
		\item  765 - 347 
	\end{enumerate}
	
	\subsection*{Método de resolución}
	Las sumas y restas en octal se hacen de manera análoga a la decimal con la salvedad de que al haber solamente 8 símbolos (0-7) cuando nos resulta un número mayor que 7 en el caso de la suma se lo deberá convertir a binario para luego obtener su equivalencia a octal tomándose de a 3 bits. Por ejemplo: $12_{10} = 1100_2 = 14_8$. \\
	En el caso de la resta se hace de manera inversa, cuando el minuendo es menor que el sustraendo y tiene que pedirle a la posición de la izquierda, este número que originalmente es octal debe convertirse al sistema decimal para poder razonar la operación de la forma a que estamos acostumbrados de manera decimal.\\ Por ejemplo: $13_8 - 6_8 = 001011_2 - 000110_2 = 11_{10} - 6_{10}$ 
	
		\begin{center}
		\begin{tabular}{ccc}
			\Suma{456}{123}{601}{a}[1~1~~] &
			\Suma{507}{265}{774}{b} [1~~]&
			\Resta{413}{256}{135}{c} [1~1~]\\ 
			\Resta{602}{375}{205}{d} [1~1~]&
			\Resta{530}{164}{344}{e} [1~1~]&
			\Resta{765}{347}{416}{f} [1~]\\
		\end{tabular}
	\end{center}
	
\begin{center}
	\begin{tabular}{cc}
		% Fila 1
		\begin{minipage}{0.48\linewidth}
			\begin{enumerate}[label=\alph*)]
				\item $6+3 = 9_{10} = 1001_2 = 11_8$ \\ $\textcolor{red}{1}+5+2 = 8_{10} = 1000_2 = 10_8$ \\ $\textcolor{red}{1}+4+1 = 6_8$	
			\end{enumerate}
		\end{minipage}
		&
		\begin{minipage}{0.48\linewidth}
			\begin{enumerate}[start=2,label=\alph*)]
				\item $7+5 = 12_{10} = 1100_2 = 14_8$ \\ $\textcolor{red}{1}+0+6 = 7_8$ \\ $5+2 = 7_8$
			\end{enumerate}
		\end{minipage}
		\\[3em]
		% Fila 2
		\begin{minipage}{0.48\linewidth}
			\begin{enumerate}[start=3,label=\alph*)]
				\item $(\textcolor{blue}{1}3_8=001011_2=11_{10}) - 6 =5$ \\ $(\textcolor{blue}{1}0_8=001000_2=8_{10})-5 = 3$ \\ $~~3-2 = 1$
			\end{enumerate}
		\end{minipage}
		&
		\begin{minipage}{0.48\linewidth}
			\begin{enumerate}[start=4,label=\alph*)]
				\item $(\textcolor{blue}{1}2_8 = 001010_2 = 10_{10}) - 5 = 5$ \\ $7_8 - 7_8 = 0$ \\ $6_8 - 3_8 = 2$
			\end{enumerate}
		\end{minipage}
		\\[3em]
		% Fila 3
		\begin{minipage}{0.48\linewidth}
			\begin{enumerate}[start=5,label=\alph*)]
				\item $(\textcolor{blue}{1}0_8=001000:2=8_{10}) -4 = 4$ \\ $(\textcolor{blue}{1}2_8 = 001010_2=10_{10})-6=4$ \\ $~~4-1 =3$
			\end{enumerate}
		\end{minipage}
		&
		\begin{minipage}{0.48\linewidth}
			\begin{enumerate}[start=6,label=\alph*)]
				\item $(\textcolor{blue}{1}5_8=001101_2=13_{10})-7=6$ \\ $5-4=1$ \\ $7-3=4$
			\end{enumerate}
		\end{minipage}
	\end{tabular}
\end{center}

		
	3. Resolver las siguientes operaciones en HEXADECIMAL (resultado y operaciones deben estar en el desarrollo).
	\begin{enumerate}
		\renewcommand{\theenumi}{\alph{enumi}}	
		\item  6A3 + 2BF
		\item  3C5 + D1A
		\item  ABC + 1DE
		\item  C89 - A1B
		\item  A4F - 8D2
		\item  F21 - E09 
	\end{enumerate}

	\subsection*{Método de resolución} 
	Las operaciones en sistema hexadecimal se hicieron de la misma forma en que se piensan en sistema decimal, con la salvedad de que en hexa se tienen 16 digitos posibles (de 0 a 9 y luego continúan con A para el número 10 hasta la F que representa el número 15). La forma de operar es la misma que en decimal aunque es importante recordar que cuando una resta no es posible y "le pide prestado al de al lado" este, le suma 16 al número solicitante respetando así, el valor posicional del sistema en que se está operando.\\ 
	
	\begin{center}
		\begin{tabular}{ccc}
			\Suma{6A3}{2BF}{962}{a} &
			\Suma{3C5}{D1A}{10DF}{b} &
			\Suma{ABC}{1DE}{C9A}{c} \\ 
			\Resta{C89}{A1B}{26E}{d} &
			\Resta{A4F}{8D2}{17D}{e} &
			\Resta{F21}{E09}{117}{f} \\
		\end{tabular}
	\end{center} 
	
	\vspace{2em} 
	
	\begin{center}
		\begin{tabular}{cc}
			% Fila 1
			\begin{minipage}{0.48\linewidth}
				\begin{enumerate}[label=\alph*)]
					\item $3+F = 12$ \\ $1+A+B=16$ \\ $1+6+2=9$	
				\end{enumerate}
			\end{minipage}
			&
			\begin{minipage}{0.48\linewidth}
				\begin{enumerate}[start=2,label=\alph*)]
					\item $5+A=F$ \\ $C+1=D$ \\ $3+D=10$
				\end{enumerate}
			\end{minipage}
			\\[3em]
			% Fila 2
			\begin{minipage}{0.48\linewidth}
				\begin{enumerate}[start=3,label=\alph*)]
					\item $C+E=A$ \\ $B+D=9$ \\ $A+1=C$
				\end{enumerate}
			\end{minipage}
			&
			\begin{minipage}{0.48\linewidth}
				\begin{enumerate}[start=4,label=\alph*)]
					\item $19-B=E$ \\ $7-1=6$ \\ $C-A=2$
				\end{enumerate}
			\end{minipage}
			\\[3em]
			% Fila 3
			\begin{minipage}{0.48\linewidth}
				\begin{enumerate}[start=5,label=\alph*)]
					\item $F-2=D$ \\ $4-D=7$ \\ $A-8=1$
				\end{enumerate}
			\end{minipage}
			&
			\begin{minipage}{0.48\linewidth}
				\begin{enumerate}[start=6,label=\alph*)]
					\item $11-9=7$ \\ $1-0=1$ \\ $F-E=1$ \\
				\end{enumerate}
			\end{minipage}
		\end{tabular}
	\end{center} 
	
	4. Realizar las siguientes operaciones aritméticas usando CA2
	\begin{enumerate}
		\renewcommand{\theenumi}{\alph{enumi}}	
		\item  17 - 7
		\item  60 - 25 
		\item  53 - 82
		\item  -23 - 25
		\item  -45 + 36
		\item  125 - 365
	\end{enumerate}
	
	\subsection*{Método de resolución}
	Para realizar las operaciones en CA2 primero se convierte a binario natural ambos operando. Luego para efectuar la suma en dicha representación, mientras que los números positivos quedan en su forma de binario natural, a los negativos se los convierte a complemento A1 invirtiendo todos sus bits, para luego al binario obtenido sumarle 1 al bit menos significativo. De esta manera se obtiene el complemento A2 y se efectúa la suma.\\
	
	\begin{table}[H]
		\centering
		\caption*{\textbf{a) Operación: 15 - 7}}
		\begin{tabular}{|l|c|c|}
			\hline
			\textbf{}       & \textbf{Operando A (15)} & \textbf{Operando B (-7)} \\
			\hline
			Binario         & \texttt{00001111}        & \texttt{00000111} \\
			Complemento A1  & ---                      & \texttt{11111000} \\
			Complemento A2  & ---                      & \texttt{11111001} \\
			\hline
			Suma (A + B)    & \multicolumn{2}{c|}{\texttt{00001000}} \\
			\hline
		\end{tabular}
	\end{table}
	
	\begin{table}[H]
		\centering
		\caption*{\textbf{b) Operación: 60 - 25}}
		\begin{tabular}{|l|c|c|}
			\hline
			\textbf{}       & \textbf{Operando A (60)} & \textbf{Operando B (-25)} \\
			\hline
			Binario         & \texttt{00111100}        & \texttt{00011001} \\
			Complemento A1  & ---                      & \texttt{11100110} \\
			Complemento A2  & ---                      & \texttt{11100111} \\
			\hline
			Suma (A + B)    & \multicolumn{2}{c|}{\texttt{00100011}} \\
			\hline
		\end{tabular}
	\end{table}
	
	\begin{table}[H]
		\centering
		\caption*{\textbf{c) Operación: 53 - 82}}
		\begin{tabular}{|l|c|c|}
			\hline
			\textbf{}       & \textbf{Operando A (53)} & \textbf{Operando B (-82)} \\
			\hline
			Binario         & \texttt{00110101}        & \texttt{01010010} \\
			Complemento A1  & ---                      & \texttt{10101101} \\
			Complemento A2  & ---                      & \texttt{10101110} \\
			\hline
			Suma (A + B)    & \multicolumn{2}{c|}{\texttt{11100011}} \\
			\hline
		\end{tabular}
	\end{table}
	
	\begin{table}[H]
		\centering
		\caption*{\textbf{d) Operación: -23 -25}}
		\begin{tabular}{|l|c|c|}
			\hline
			\textbf{}       & \textbf{Operando A (-23)} & \textbf{Operando B (-25)} \\
			\hline
			Binario         & \texttt{00010111}        & \texttt{00011001} \\
			Complemento A1  & \texttt{11101000}        & \texttt{11100110} \\
			Complemento A2  & \texttt{11101001}        & \texttt{11100111} \\
			\hline
			Suma (A + B)    & \multicolumn{2}{c|}{\texttt{11010000}} \\
			\hline
		\end{tabular}
	\end{table}
	
	\begin{table}[H]
		\centering
		\caption*{\textbf{e) Operación: -45 + 36}}
		\begin{tabular}{|l|c|c|}
			\hline
			\textbf{}       & \textbf{Operando A (-45)} & \textbf{Operando B (36)} \\
			\hline
			Binario         & \texttt{00101101}        & \texttt{00100100} \\
			Complemento A1  & \texttt{11010010}        & ---               \\
			Complemento A2  & \texttt{11010011}        & ---               \\
			\hline
			Suma (A + B)    & \multicolumn{2}{c|}{\texttt{11110111}} \\
			\hline
		\end{tabular}
	\end{table}
	
	\begin{table}[H]
		\centering
		\caption*{\textbf{f) Operación: 125 - 365}}
		\begin{tabular}{|l|c|c|}
			\hline
			\textbf{}       & \textbf{Operando A (125)} & \textbf{Operando B (-365)} \\
			\hline
			Binario         & \texttt{001111101}        & \texttt{101101101} \\
			Complemento A1  & ---                      & \texttt{010010010} \\
			Complemento A2  & ---                      & \texttt{010010011} \\
			\hline
			Suma (A + B)    & \multicolumn{2}{c|}{\texttt{100010000}} \\
			\hline
		\end{tabular}
	\end{table}
	
		\vspace{3em} 
	
	5. Teniendo en cuenta que los códigos de Gray tienen una distancia de 1 bit entre cada uno de sus valores, cree una secuencia de 4 bits que cumpla con las siguientes consignas:
	\begin{enumerate}
		\renewcommand{\theenumi}{\alph{enumi}}	
		\item  Debe tener distancia de 1 bit entre cada uno de sus valores.
		\item  El primer y último valor de la lista también debe tener una distancia de 1.
	\end{enumerate}
	
	\vspace{2em} 
	
	\subsection*{Método de conversión}
	El código reflejado de Gray se obtiene aplicando la operación Xor bit a bit entre el número en binario y el mismo número desplazado un bit hacia la derecha, consiguiendo así un sistema de numeración binario donde dos números consecutivos difieren en un solo bit.
	
	\begin{center}
		\begin{tabular}{cccc}
			\Xor{0000}{0000}{0000}[0] &
			\Xor{0001}{0000}{0001}[1] &
			\Xor{0010}{0001}{0011}[2] &
			\Xor{0011}{0001}{0010}[3] \\
			\\[0.5em]
			\Xor{0100}{0010}{0110}[4] &
			\Xor{0101}{0010}{0111}[5] &
			\Xor{0110}{0011}{0101}[6] &
			\Xor{0111}{0011}{0100}[7] \\
			\\[0.5em]
			\Xor{1000}{0100}{1100}[8] &
			\Xor{1001}{0100}{1101}[9] &
			\Xor{1010}{0101}{1111}[10] &
			\Xor{1011}{0101}{1110}[11] \\
			\\[0.5em]
			\Xor{1100}{0110}{1010}[12] &
			\Xor{1101}{0110}{1011}[13] &
			\Xor{1110}{0111}{1001}[14] &
			\Xor{1111}{0111}{1000}[15] \\
		\end{tabular}
	\end{center}
	
	
	
	\begin{table}[h!]
		\centering
		\large
		\begin{tabular}{|c|c|c|}
			\hline
			Decimal & Binario & Código Gray \\
			\hline
			0  & 0000 & 0000 \\
			1  & 0001 & 0001 \\
			2  & 0010 & 0011 \\
			3  & 0011 & 0010 \\
			4  & 0100 & 0110 \\
			5  & 0101 & 0111 \\
			6  & 0110 & 0101 \\
			7  & 0111 & 0100 \\
			8  & 1000 & 1100 \\
			9  & 1001 & 1101 \\
			10 & 1010 & 1111 \\
			11 & 1011 & 1110 \\
			12 & 1100 & 1010 \\
			13 & 1101 & 1011 \\
			14 & 1110 & 1001 \\
			15 & 1111 & 1000 \\
			\hline
		\end{tabular}
		
	\end{table}



	
\end{document}
\grid
